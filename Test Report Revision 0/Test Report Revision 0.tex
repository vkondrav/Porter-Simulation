%%%%%%%%%%%%%%%%%%%%%%%%%%%%%%%%%%%%%%%%%%%%%%%%%%%%%%%%%%%%%%%%%%%%%%
% LaTeX Example: Project Report
%
% Source: http://www.howtotex.com
%
% Feel free to distribute this example, but please keep the referral
% to howtotex.com
% Date: March 2011 
% 
%%%%%%%%%%%%%%%%%%%%%%%%%%%%%%%%%%%%%%%%%%%%%%%%%%%%%%%%%%%%%%%%%%%%%%
% How to use writeLaTeX: 
%
% You edit the source code here on the left, and the preview on the
% right shows you the result within a few seconds.
%
% Bookmark this page and share the URL with your co-authors. They can
% edit at the same time!
%
% You can upload figures, bibliographies, custom classes and
% styles using the files menu.
%
% If you're new to LaTeX, the wikibook is a great place to start:
% http://en.wikibooks.org/wiki/LaTeX
%
%%%%%%%%%%%%%%%%%%%%%%%%%%%%%%%%%%%%%%%%%%%%%%%%%%%%%%%%%%%%%%%%%%%%%%
% Edit the title below to update the display in My Documents
%\title{Project Report}
%
%%% Preamble
\documentclass[paper=letter, fontsize=10pt]{scrartcl}
\usepackage[body={7in,7.5in},top=1in, bottom=1in]{geometry}
\usepackage[T1]{fontenc}
\usepackage{fourier}

\usepackage[english]{babel}															% English language/hyphenation
\usepackage[protrusion=true,expansion=true]{microtype}	
\usepackage{amsmath,amsfonts,amsthm} % Math packages
\usepackage[pdftex]{graphicx}	
\usepackage{url}
\usepackage{enumerate}
\usepackage{lastpage}


%%% Custom sectioning
\usepackage{sectsty}
\allsectionsfont{\normalfont\scshape}


%%% Custom headers/footers (fancyhdr package)
\usepackage{fancyhdr}
\pagestyle{fancy}
\fancyhead[L]{}
\fancyhead[c]{Requirements Revision 0}
\fancyhead[R]{\today}											
\fancyfoot[L]{}											 
\fancyfoot[C]{}											
\fancyfoot[R]{\thepage\ of \pageref{LastPage}}		% Pagenumbering
\renewcommand{\headrulewidth}{0.4pt}				% Remove header underlines
\renewcommand{\footrulewidth}{0.4pt}				% Remove footer underlines
\setlength{\headheight}{13.6pt}


%%% Equation and float numbering
\numberwithin{equation}{section}		% Equationnumbering: section.eq#
\numberwithin{figure}{section}			% Figurenumbering: section.fig#
\numberwithin{table}{section}				% Tablenumbering: section.tab#


%%% Maketitle metadata
\newcommand{\horrule}[1]{\rule{\linewidth}{#1}} 	% Horizontal rule
\newcommand{\ts}{\textsuperscript}

%%% Begin document
\begin{document}

\begin{titlepage}

\newcommand{\HRule}{\rule{\linewidth}{0.5mm}} % Defines a new command for the horizontal lines, change thickness here
\newcommand{\authors}{\shortstack{Vitaliy Kondratiev,\\Nathan Johrendt,\\Tyler Lyn,\\Mark Gammie}}

\begin{center}
 
%----------------------------------------------------------------------------------------
%	HEADING SECTIONS
%----------------------------------------------------------------------------------------

\textsc{\LARGE McMaster University}\\[1.5cm] % Name of your university/college
\textsc{\Large CAS 4ZP6}\\[0.5cm]
\textsc{\Large Team 9} \\[0.5cm]
\textsc{\Large Capstone Project 2013/2014}\\[0.5cm] % Major heading such as course name
\textsc{\large Porter Simulation}\\[0.5cm] % Minor heading such as course title

%----------------------------------------------------------------------------------------
%	TITLE SECTION
%----------------------------------------------------------------------------------------

\HRule \\[0.4cm]
{ \huge \bfseries Test Report Revision 0}\\[0.4cm] % Title of your document
\HRule \\[1.5cm]
 
%----------------------------------------------------------------------------------------
%	AUTHOR SECTION
%----------------------------------------------------------------------------------------

\begin{minipage}{0.4\textwidth}
\begin{flushleft} \large
\emph{Authors:}\\
Vitaliy Kondratiev - 0945220\\
Nathan Johrendt - 0950519\\
Tyler Lyn - 0948978\\
Mark Gammie - 0964156
\end{flushleft}
\end{minipage}
~
\begin{minipage}{0.4\textwidth}
\begin{flushright} \large
\emph{Supervisor:} \\
Dr. Douglas Down % Supervisor's Name
\end{flushright}
\end{minipage}\\[4cm]

% If you don't want a supervisor, uncomment the two lines below and remove the section above
%\Large \emph{Author:}\\
%John \textsc{Smith}\\[3cm] % Your name

%----------------------------------------------------------------------------------------
%	DATE SECTION
%----------------------------------------------------------------------------------------

{\large \today}\\[3cm] % Date, change the \today to a set date if you want to be precise

%----------------------------------------------------------------------------------------
%	LOGO SECTION
%----------------------------------------------------------------------------------------

%\includegraphics{Logo}\\[1cm] % Include a department/university logo - this will require the graphicx package
 
%----------------------------------------------------------------------------------------
%Template taken from: http://www.softwaretestinghelp.com/test-plan-sample-softwaretesting-and-quality-assurance-templates/

\vfill % Fill the rest of the page with whitespace
\end{center}
\end{titlepage}

\setcounter{tocdepth}{2}

\tableofcontents

\newpage

\section{Revision History}
\begin{center}
    \begin{tabular}{| c | l | l | l |}
    \hline
    Revision \# & Author & Date & Comment \\ \hline
  	1 & \shortstack{\\Nathan Johrendt} & March 17 & Test Report Template Complete \\ \hline
    \end{tabular}
\end{center}

\section{Introduction}
%Intro, refer to test plan, manual testing ommitted

\section{System Test Reports}
%Specific system tests summarized in terms of initial state, input and expected output. Traceability to requirements and modules
\subsection{Input/Initialization Correctness 1}
Once input values are loaded by the simulation, it has become initialized with modified values. Before beginning simulation, the simulation will re-print the information it just imported to confirm that it has found the right information.
\begin{enumerate}[(i)]
	\item \textbf{Initial State:} Uninitialized Simulation   
	\item \textbf{Input:} Simulation input variables from interface, as well as a file location for both data.csv and schedule.csv for importing job data and porter shift information respectively.
%Input screenshot
	\item \textbf{Expected Output:} Simulation outputs variable data identical to that defined by the interface.
	\item \textbf{Actual Output:} 
%Output Values just text dump
\end{enumerate}

\subsection{Event List Correctness 1}
As the simulation runs, when an event is dispatched to a porter, it will then have fewer jobs left to assign. With 30 porters on a 24 hour schedule, no jobs will remain undone, and any that remain will represent an error to examine.
\begin{enumerate}[(i)]
	\item \textbf{Initial State:} Simulation event list is generated by initializing the simulation through the interface
	\item \textbf{Input:} Event list is generated from jobs stored in the data.csv file. The simulation will be configured to add jobs for one day, with 30 porters scheduled 24 hours a day.
	\item \textbf{Expected Output:} Conclude that no incomplete jobs remain
	\item \textbf{Actual Output:}
\end{enumerate}

\subsection{State Change Correctness 1}
With only five Porters working, they will be required to change state frequently as they complete all of a regular day's tasks understaffed. Any problems, beyond the obvious delays, will result in a Porter getting stuck with a job at a particular state. This will be recorded in the output data and can be addressed.
\begin{enumerate}[(i)]
	\item \textbf{Initial State:} Simulation event list is generated by initializing the simulation through the interface
	\item \textbf{Input:} Event list is generated from jobs stored in the data.csv file. Porter schedule is generated from the schedule.csv file. The simulation will be configured to add jobs for one day, with five porters scheduled 24 hours a day.
	\item \textbf{Expected Output:} All Porters complete jobs dispatched to them successfully
	\item \textbf{Actual Output:}
\end{enumerate}

\subsection{Porter/Event Linkage Correctness 1}
Every time an Event moves from pending to dispatched, a Porter must be linked to that event. This test is to ensure that all pairings of job and Porter are unique, and that after completing a job, a porter will return to "available" and continue accepting new jobs.
\begin{enumerate}[(i)]
	\item \textbf{Initial State:} Simulation event list and porter schedule are generated by initializing the simulation through the interface
	\item \textbf{Input:} Event list is generated from jobs stored in the data.csv file. Porter schedule is generated from the schedule.csv file. The simulation will be configured to add jobs for one day, with 30 porters scheduled 24 hours a day.
	\item \textbf{Expected Output:} Porter and Event are linked together uniquely
	\item \textbf{Actual Output:}
\end{enumerate}

\subsection{Task Pool Correctness 1}
With only a single porter active, the majority of the jobs added by the simulation will simply be added to the dispatcher and wait. This test is to ensure that the dispatcher holds incomplete jobs correctly and that they are reported after the simulation completes.
\begin{enumerate}[(i)]
	\item \textbf{Initial State:} Simulation event list and porter schedule are generated by initializing the simulation through the interface
	\item \textbf{Input:} Event list is generated from jobs stored in the data.csv file. Porter schedule is generated from the schedule.csv file. The simulation will be configured to add jobs for one day, with one Porter scheduled 24 hours a day.
	\item \textbf{Expected Output:} Simulation output properly stores many incomplete job entries.
	\item \textbf{Actual Output:}
\end{enumerate}

\subsection{Termination Correctness 1}
After accepting all inputs and initializing, the simulation will compute five days of operational data and output the results.
\begin{enumerate}[(i)]
	\item \textbf{Initial State:} Simulation is initialized through the interface
	\item \textbf{Input:} Event list is generated from jobs stored in the data.csv file. Porter schedule is generated from the schedule.csv file. The simulation will be configured to add jobs for five days, with 30 porters scheduled 24 hours a day.
	\item \textbf{Expected Output:} Simulation outputs that it has completed each step and has written the results to the dashboard.
	\item \textbf{Actual Output:}
\end{enumerate}
\subsection{The Golden Test}
Using a 'gold copy' of a set of events and Porter shifts, new program builds are retested using the 'gold copies' to ensure consistency of execution by examining inconsistencies in the output file. This specific set of events is maintained and retested on new revisions of the software. A difference file can then be created in excel comparing 'gold copy' statistics against the newly tested code outlining the inconsistencies between them. The maximum variance allowed in the results is still being determined. 
\begin{enumerate}[(i)]
	\item \textbf{Initial State:} Simulation is initialized through the interface
	\item \textbf{Input:} 'Gold Copy' set of events imported from golddata.csv, as well as Porter shift information from goldschedule.csv 
	\item \textbf{Expected Output:} Simulation outputs that it has completed each step and has written the results to the dashboard.
	\item \textbf{Actual Output:}
\end{enumerate}
\subsection{Compatibility Test 1}
Simulation is run and compared to results computed on other operating systems to ensure it is functioning normally in the new environment
\begin{enumerate}[(i)]
	\item \textbf{Initial State:} Unknown Operating System with the simulation accessible on local storage
	\item \textbf{Input:} 'Gold Copy' simulation parameters (details outline previously under 'The Golden Test')
	\item \textbf{Expected Output:} Simulated results are consistent with previously generated values
	\item \textbf{Actual Output:}
\end{enumerate}

\section{Nonfunctional Test Reports}
%Usability, Performance and Robustness. Note: Write a new test case for performance, stakeholders wanted 'reasonable' time for execution.
\subsection{Usability Test 1}
End user is provided with a set of instructions from the user manual on how to initialize and run the simulation. The success of this test is determined by how much external assistance the End User requires from the development team on their first use of the software. 
\begin{enumerate}[(i)]
	\item \textbf{Initial State:} Simulation prior to execution
	\item \textbf{Input:} End user with a copy of both the simulation software package and accompanying user manual.
	\item \textbf{Expected Output:} The end user successfully Initializes the Simulation
	\item \textbf{Actual Output:}
\end{enumerate}

\subsection{Performance Test 1}
The project stakeholders have placed a loose time limit on the duration of execution for a single simulation run of fifteen minutes. No correct simulation execution has taken longer than five minutes, with most between one and two minutes, but ensuring this general timing on each system the simulation is run on is the purpose of this test.
\begin{enumerate}[(i)]
	\item \textbf{Initial State:} Simulation prior to execution 
	\item \textbf{Input:} Any combination of inputs that the interface will accept and begin simulating
	\item \textbf{Expected Output:} The simulation completes execution taking between thirty seconds and five minutes 
	\item \textbf{Actual Output:}
\end{enumerate}

\section{Summary}
%Changes made as a result of testing, talk about automated testing

\section{Figures and Tables Appendix}
\begin{enumerate}[(a)]
	\item Figure 3.1: 
\end{enumerate}

%%% End document
\end{document}