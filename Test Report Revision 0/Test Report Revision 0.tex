%%%%%%%%%%%%%%%%%%%%%%%%%%%%%%%%%%%%%%%%%%%%%%%%%%%%%%%%%%%%%%%%%%%%%%
% LaTeX Example: Project Report
%
% Source: http://www.howtotex.com
%
% Feel free to distribute this example, but please keep the referral
% to howtotex.com
% Date: March 2011 
% 
%%%%%%%%%%%%%%%%%%%%%%%%%%%%%%%%%%%%%%%%%%%%%%%%%%%%%%%%%%%%%%%%%%%%%%
% How to use writeLaTeX: 
%
% You edit the source code here on the left, and the preview on the
% right shows you the result within a few seconds.
%
% Bookmark this page and share the URL with your co-authors. They can
% edit at the same time!
%
% You can upload figures, bibliographies, custom classes and
% styles using the files menu.
%
% If you're new to LaTeX, the wikibook is a great place to start:
% http://en.wikibooks.org/wiki/LaTeX
%
%%%%%%%%%%%%%%%%%%%%%%%%%%%%%%%%%%%%%%%%%%%%%%%%%%%%%%%%%%%%%%%%%%%%%%
% Edit the title below to update the display in My Documents
%\title{Project Report}
%
%%% Preamble
\documentclass[paper=letter, fontsize=10pt]{scrartcl}
\usepackage[body={7in,7.5in},top=1in, bottom=1in]{geometry}
\usepackage[T1]{fontenc}
\usepackage{fourier}

\usepackage[english]{babel}															% English language/hyphenation
\usepackage[protrusion=true,expansion=true]{microtype}	
\usepackage{amsmath,amsfonts,amsthm} % Math packages
\usepackage[pdftex]{graphicx}	
\usepackage{url}
\usepackage{enumerate}
\usepackage{lastpage}


%%% Custom sectioning
\usepackage{sectsty}
\allsectionsfont{\normalfont\scshape}


%%% Custom headers/footers (fancyhdr package)
\usepackage{fancyhdr}
\pagestyle{fancy}
\fancyhead[L]{}
\fancyhead[c]{Requirements Revision 0}
\fancyhead[R]{\today}											
\fancyfoot[L]{}											 
\fancyfoot[C]{}											
\fancyfoot[R]{\thepage\ of \pageref{LastPage}}		% Pagenumbering
\renewcommand{\headrulewidth}{0.4pt}				% Remove header underlines
\renewcommand{\footrulewidth}{0.4pt}				% Remove footer underlines
\setlength{\headheight}{13.6pt}


%%% Equation and float numbering
\numberwithin{equation}{section}		% Equationnumbering: section.eq#
\numberwithin{figure}{section}			% Figurenumbering: section.fig#
\numberwithin{table}{section}				% Tablenumbering: section.tab#


%%% Maketitle metadata
\newcommand{\horrule}[1]{\rule{\linewidth}{#1}} 	% Horizontal rule
\newcommand{\ts}{\textsuperscript}

%%% Begin document
\begin{document}

\begin{titlepage}

\newcommand{\HRule}{\rule{\linewidth}{0.5mm}} % Defines a new command for the horizontal lines, change thickness here
\newcommand{\authors}{\shortstack{Vitaliy Kondratiev,\\Nathan Johrendt,\\Tyler Lyn,\\Mark Gammie}}

\begin{center}
 
%----------------------------------------------------------------------------------------
%	HEADING SECTIONS
%----------------------------------------------------------------------------------------

\textsc{\LARGE McMaster University}\\[1.5cm] % Name of your university/college
\textsc{\Large CAS 4ZP6}\\[0.5cm]
\textsc{\Large Team 9} \\[0.5cm]
\textsc{\Large Capstone Project 2013/2014}\\[0.5cm] % Major heading such as course name
\textsc{\large Porter Simulation}\\[0.5cm] % Minor heading such as course title

%----------------------------------------------------------------------------------------
%	TITLE SECTION
%----------------------------------------------------------------------------------------

\HRule \\[0.4cm]
{ \huge \bfseries Test Report Revision 0}\\[0.4cm] % Title of your document
\HRule \\[1.5cm]
 
%----------------------------------------------------------------------------------------
%	AUTHOR SECTION
%----------------------------------------------------------------------------------------

\begin{minipage}{0.4\textwidth}
\begin{flushleft} \large
\emph{Authors:}\\
Vitaliy Kondratiev - 0945220\\
Nathan Johrendt - 0950519\\
Tyler Lyn - 0948978\\
Mark Gammie - 0964156
\end{flushleft}
\end{minipage}
~
\begin{minipage}{0.4\textwidth}
\begin{flushright} \large
\emph{Supervisor:} \\
Dr. Douglas Down % Supervisor's Name
\end{flushright}
\end{minipage}\\[4cm]

% If you don't want a supervisor, uncomment the two lines below and remove the section above
%\Large \emph{Author:}\\
%John \textsc{Smith}\\[3cm] % Your name

%----------------------------------------------------------------------------------------
%	DATE SECTION
%----------------------------------------------------------------------------------------

{\large \today}\\[3cm] % Date, change the \today to a set date if you want to be precise

%----------------------------------------------------------------------------------------
%	LOGO SECTION
%----------------------------------------------------------------------------------------

%\includegraphics{Logo}\\[1cm] % Include a department/university logo - this will require the graphicx package
 
%----------------------------------------------------------------------------------------
%Template taken from: http://www.softwaretestinghelp.com/test-plan-sample-softwaretesting-and-quality-assurance-templates/

\vfill % Fill the rest of the page with whitespace
\end{center}
\end{titlepage}

\setcounter{tocdepth}{2}

\tableofcontents

\newpage

\section{Revision History}
\begin{center}
    \begin{tabular}{| c | l | l | l |}
    \hline
    Revision \# & Author & Date & Comment \\ \hline
  	1 & \shortstack{\\Nathan Johrendt} & March 17 & Test Report Template Complete \\ \hline
  	2 & \shortstack{\\Vitaliy Kondratiev,\\Nathan Johrendt,\\Tyler Lyn,\\Mark Gammie} & March 17 & Test Plan Updates \\ \hline
  	3 & \shortstack{\\Vitaliy Kondratiev,\\Nathan Johrendt,\\Tyler Lyn,\\Mark Gammie} & March 18 & Test Plan Updates \\ \hline
    \end{tabular}
\end{center}

\section{Introduction}
This testing report is the realization of the test plan written for the Porter Simulation. It contains descriptions and output data for the most important tests from the test plan. The majority of the following test cases are dynamic; many have static versions found within the test plan, but their results are not as valuable, and so they have been omitted.

\section{System Test Reports}
%Specific system tests summarized in terms of initial state, input and expected output. Traceability to requirements and modules
\subsection{Input/Initialization Correctness 1}
Once input values are loaded by the simulation, it has become initialized with modified values. Before beginning simulation, the simulation will re-print the information it just imported to confirm that it has found the right information. A simple check to ensure data is being handled correctly during import. While there are not many variables to be tested now, several will be added before the final implementation is complete. This test deals primarily with the Input Module which traces to requirements section 8.1.
\begin{enumerate}[(i)]
	\item \textbf{Initial State:} Uninitialized Simulation   
	\item \textbf{Input:} Simulation input variables from interface, as well as a file location for both data.csv and schedule.csv for importing job data and porter shift information respectively. See Figure 3.1: User Interface below.
	\begin{figure}[!htbp]
		\begin{center}
			\includegraphics[width=1\columnwidth, height=0.5\textheight, keepaspectratio]{Interface.png}
			\caption{User Interface}
		\end{center}
	\end{figure}
	\item \textbf{Expected Output:} Simulation outputs variable data identical to that defined by the interface. 
	\item \textbf{Actual Output:} duration = 86400.0 \\
	 statsource = "C:$\backslash$ Users$\backslash$ Nathan$\backslash$ Documents$\backslash$ GitHub$\backslash$ Porter-Simulation$\backslash$ Simulation Prototype" \\
	 shiftsource = "C:$\backslash$ Users$\backslash$ Nathan$\backslash$ Documents$\backslash$ GitHub$\backslash$ Porter-Simulation$\backslash$ Simulation Prototype" \\
	 outputloc = "C:$\backslash$ Users$\backslash$ Nathan$\backslash$ Documents$\backslash$ GitHub$\backslash$ Porter-Simulation$\backslash$ Simulation Prototype" \\
\end{enumerate}

\subsection{Event List Correctness 1}
As the simulation runs, when an event is dispatched to a porter, it will then have fewer jobs left to assign. With 30 porters on a 24 hour schedule, no jobs will remain undone, and any that remain will represent an error to examine. This test involves the simulation core, dispatch and export modules.
\begin{enumerate}[(i)]
	\item \textbf{Initial State:} Simulation event list is generated by initializing the simulation through the interface
	\item \textbf{Input:} Event list is generated from jobs stored in the data.csv file. The simulation will be configured to add jobs for one day, with 30 porters scheduled 24 hours a day. See Table 3.1: Porter Schedule in schedule.csv below.
	\begin{table}
	\caption{Porter Shift Information in schedule.csv for Event List Correctness 1}
	\begin{center}
    	\begin{tabular}{| c | l | l | l | l |}
    		\hline
        	Shift ID & StartTime & EndTime & Porter Ids & Day \\ \hline
  			0 & 12:00 AM & 11:59 PM & 0,1,2,3,4,5,6,7,8,9,10,11,12,13,14,15,16,17,18,19,20,21,22,23,24,25,26,27,28,29 & 0 \\ \hline
  			1 & 12:00 AM & 11:59 PM & 0,1,2,3,4,5,6,7,8,9,10,11,12,13,14,15,16,17,18,19,20,21,22,23,24,25,26,27,28,29 & 1 \\ \hline
  			2 & 12:00 AM & 11:59 PM & 0,1,2,3,4,5,6,7,8,9,10,11,12,13,14,15,16,17,18,19,20,21,22,23,24,25,26,27,28,29 & 2 \\ \hline
  			3 & 12:00 AM & 11:59 PM & 0,1,2,3,4,5,6,7,8,9,10,11,12,13,14,15,16,17,18,19,20,21,22,23,24,25,26,27,28,29 & 3 \\ \hline
  			4 & 12:00 AM & 11:59 PM & 0,1,2,3,4,5,6,7,8,9,10,11,12,13,14,15,16,17,18,19,20,21,22,23,24,25,26,27,28,29 & 4 \\ \hline
  			5 & 12:00 AM & 11:59 PM & 0,1,2,3,4,5,6,7,8,9,10,11,12,13,14,15,16,17,18,19,20,21,22,23,24,25,26,27,28,29 & 5 \\ \hline
  			6 & 12:00 AM & 11:59 PM & 0,1,2,3,4,5,6,7,8,9,10,11,12,13,14,15,16,17,18,19,20,21,22,23,24,25,26,27,28,29 & 6 \\ \hline
    	\end{tabular}
	\end{center}
	\end{table}
	\item \textbf{Expected Output:} Conclude that no incomplete jobs remain
	\item \textbf{Actual Output:} \\
	Job145: E2 Orthopedics Room 07 Bed 03 -> E3 Nurse Station (3497) \\
Porter28 is dispatched Job147: Emergency Department - GRIDLOCK (2286) -> E3 Nurse Station (3497) \\
Porter28 is in-progress Job147: Emergency Department - GRIDLOCK (2286) -> E3 Nurse Station (3497) \\
Porter20 is complete Job148: Emergency Department - GRIDLOCK (2286) -> F4 Surg Oncology (3501) \\
Job144: E2 Orthopedics Room 15 Bed 02 -> E3 Nurse Station (3497) \\
Porter29 is dispatched Job145: E2 Orthopedics Room 07 Bed 03 -> E3 Nurse Station (3497) \\
Porter1 is pending \\
Porter29 is in-progress Job145: E2 Orthopedics Room 07 Bed 03 -> E3 Nurse Station (3497)\\
Porter24 is complete Job147: Emergency Department - GRIDLOCK (2286) -> E3 Nurse Station (3497)\\
Job150: Emergency Department Bed 01 -> F4 Surg Oncology (3501)\\
Porter24 is dispatched Job144: E2 Orthopedics Room 15 Bed 02 -> E3 Nurse Station (3497)\\
Porter28 is in-progress Job144: E2 Orthopedics Room 15 Bed 02 -> E3 Nurse Station (3497)\\
Porter24 is pending\\
Porter28 is complete Job144: E2 Orthopedics Room 15 Bed 02 -> E3 Nurse Station (3497)\\
Porter29 is complete Job145: E2 Orthopedics Room 07 Bed 03 -> E3 Nurse Station (3497)\\
Porter25 is pending\\
Porter28 is pending\\
Porter29 is pending\\
*****SIMULATION COMPLETE*****\\
\end{enumerate}

\subsection{State Change Correctness 1}
With only five Porters working, they will be required to change state frequently as they complete all of a regular day's tasks understaffed. Any problems, beyond the obvious delays, will result in a Porter getting stuck with a job at a particular state. This will be recorded in the output data and can be addressed. This test deals exclusively with the simulation core, dispatch and export modules.
\begin{enumerate}[(i)]
	\item \textbf{Initial State:} Simulation event list is generated by initializing the simulation through the interface
	\item \textbf{Input:} Event list is generated from jobs stored in the data.csv file. Porter schedule is generated from the schedule.csv file. The simulation will be configured to add jobs for one day, with five porters scheduled 24 hours that day.
	\begin{table}
	\caption{Porter Shift Information in schedule.csv for State Change Correctness 1}
	\begin{center}
    	\begin{tabular}{| c | l | l | l | l |}
    		\hline
        	Shift ID & StartTime & EndTime & Porter Ids & Day \\ \hline
  			0 & 12:00 AM & 11:59 PM & 0,1,2,3,4 & 0 \\ \hline
    	\end{tabular}
	\end{center}
	\end{table}
	\item \textbf{Expected Output:} All Porters complete jobs dispatched to them successfully until the simulation duration expires
	\item \textbf{Actual Output:}\\
	Job6: CDU Express Bed 05 -> X-Ray Core - B1 (2245)\\
Job10: Ultrasound (2276) -> F3 Medicine Room 12 Bed 02\\
Porter2 is dispatched Job6: CDU Express Bed 05 -> X-Ray Core - B1 (2245)\\
Porter2 is in-progress Job6: CDU Express Bed 05 -> X-Ray Core - B1 (2245)\\
Porter1 is complete Job3: X-Ray Core - B1 (2245) -> C3 Oncology Room 26 Bed 01\\
Porter4 is pending\\
Porter0 is complete Job4: F4 Surg Oncology Room 12 Bed 02 -> Ultrasound (2276)\\
Porter2 is complete Job6: CDU Express Bed 05 -> X-Ray Core - B1 (2245)\\
Job38: E3 Medicine Room 02 Bed 04 -> MRI (2778)\\
Porter4 is dispatched Job10: Ultrasound (2276) -> F3 Medicine Room 12 Bed 02\\
Job37: Ultrasound (2276) -> C3 Oncology Room 31 Bed 01\\
Porter3 is dispatched Job38: E3 Medicine Room 02 Bed 04 -> MRI (2778)\\
Job33: C3 Oncology Room 07 Bed 01 -> X-Ray Core - B1 (2245)\\
Porter1 is dispatched Job37: Ultrasound (2276) -> C3 Oncology Room 31 Bed 01\\
Job39: F4 Surg Oncology Room 05 Bed 02 -> PACU (3415)\\
Porter0 is dispatched Job33: C3 Oncology Room 07 Bed 01 -> X-Ray Core - B1 (2245)\\
Porter3 is in-progress Job38: E3 Medicine Room 02 Bed 04 -> MRI (2778)\\
*****SIMULATION COMPLETE*****\\
\end{enumerate}

\subsection{State Change Correctness 3}
\begin{enumerate}[(i)]
		\item \textbf{Initial State:} Initialized simulation
		\item \textbf{Input:} porterUnit.py unit testing file
		\item \textbf{Description:} Through the use of UnitTest in python, this test checks that a porter starts in a pending state and is able to iterates through all states when a porter is given a job.
		\item \textbf{Expected Output:} The porter is in the pending state after work is complete.
		\item \textbf{Actual Output:} Run the python module: python porterUnit.py\\
.\\
----------------------------------------------------------------------\\
Ran 1 test in 0.001s\\

OK
\end{enumerate}

\subsection{Porter/Event Linkage Correctness 1}
Every time an Event moves from pending to dispatched, a Porter must be linked to that event. This test is to ensure that all pairings of job and Porter are unique, and that after completing a job, a porter will return to "available" and continue accepting new jobs. This test involves the simulation core, dispatch and export modules.
\begin{enumerate}[(i)]
	\item \textbf{Initial State:} Simulation event list and porter schedule are generated by initializing the simulation through the interface
	\item \textbf{Input:} Event list is generated from jobs stored in the data.csv file. Porter schedule is generated from the schedule.csv file. The simulation will be configured to add jobs for one day, with 30 porters scheduled 24 hours that day.
	\begin{table}
	\caption{Porter Shift Information in schedule.csv for Porter/Event Linkage Correctness 1}
	\begin{center}
    	\begin{tabular}{| c | l | l | l | l |}
    		\hline
        	Shift ID & StartTime & EndTime & Porter Ids & Day \\ \hline
  			0 & 12:00 AM & 11:59 PM & 0,1,2,3,4,5,6,7,8,9,10,11,12,13,14,15,16,17,18,19,20,21,22,23,24,25,26,27,28,29 & 0 \\ \hline
    	\end{tabular}
	\end{center}
	\end{table}
	\item \textbf{Expected Output:} Porter and Event are linked together uniquely
	\item \textbf{Actual Output:} 
Porter23 is complete Job13: CDU Express Bed 05 -> Ultrasound (2276)\\
Porter17 is pending\\
Porter27 is complete Job12: E3 Medicine Room 02 Bed 01 -> Ultrasound (2276)\\
Porter21 is pending\\
Porter26 is pending\\
Porter24 is pending\\
Porter28 is complete Job18: CT - B1 (3023) -> E3 Medicine Room 01 Bed 01
Porter25 is complete Job26: Ultrasound (2276) -> E3 Medicine Room 02 Bed 01
Porter23 is pending\\
Porter29 is complete Job24: Ultrasound (2276) -> CDU Express Bed 05
Porter27 is pending\\
Porter25 is pending\\
Porter28 is pending\\
Porter29 is pending\\
*****SIMULATION COMPLETE*****\\
\end{enumerate}

\subsection{Task Pool Correctness 1}
With only a single porter active, the majority of the jobs added by the simulation will simply be added to the dispatcher and wait. This test is to ensure that the dispatcher holds incomplete jobs correctly and that they are reported after the simulation completes. This test involves the simulation core, dispatch and export modules.
\begin{enumerate}[(i)]
	\item \textbf{Initial State:} Simulation event list and porter schedule are generated by initializing the simulation through the interface
	\item \textbf{Input:} Event list is generated from jobs stored in the data.csv file. Porter schedule is generated from the schedule.csv file. The simulation will be configured to add jobs for one day, with one Porter scheduled 24 hours a day.
	\begin{table}
	\caption{Porter Shift Information in schedule.csv for Task Pool Correctness 1}
	\begin{center}
    	\begin{tabular}{| c | l | l | l | l |}
    		\hline
        	Shift ID & StartTime & EndTime & Porter Ids & Day \\ \hline
  			0 & 12:00 AM & 11:59 PM & 0 & 0 \\ \hline
    	\end{tabular}
	\end{center}
	\end{table}
	\item \textbf{Expected Output:} Simulation output properly stores many incomplete job entries.
	\item \textbf{Actual Output:} %Need Output Here
\end{enumerate}

\subsection{Task Pool Correctness 3}
Through the use of UnitTest in python, this test checks that the dispatcher correctly adds a new job to the Task Pool if there are no available Porters. Three specific variables are tested for: Priority Weight, Appointment Factor and the length of the pending jobs list. This test deals exclusively with the dispatcher module.
\begin{enumerate}[(i)]
	\item \textbf{Initial State:} Initialized simulation
	\item \textbf{Input:} dispatcherUnit.py unit testing file %Could include source code here
	\item \textbf{Expected Output:} Conclude that the Task Pool contains one additional event as the number of pending jobs has increased by one.
	\item \textbf{Actual Output:} Run the python module: python dispatcherUnit.py\\
...\\
----------------------------------------------------------------------\\
Ran 3 tests in 0.003s\\

OK
\end{enumerate}

\subsection{Termination Correctness 1}
After accepting all inputs and initializing, the simulation will compute five days of operational data and output the results. This test involves all modules.
\begin{enumerate}[(i)]
	\item \textbf{Initial State:} Simulation is initialized through the interface
	\item \textbf{Input:} Event list is generated from jobs stored in the data.csv file. Porter schedule is generated from the schedule.csv file. The simulation will be configured to add jobs for five days, with 30 porters scheduled 24 hours a day.
	\begin{table}
	\caption{Porter Shift Information in schedule.csv for Termination Correctness 1}
	\begin{center}
    	\begin{tabular}{| c | l | l | l | l |}
    		\hline
        	Shift ID & StartTime & EndTime & Porter Ids & Day \\ \hline
  			0 & 12:00 AM & 11:59 PM & 0,1,2,3,4,5,6,7,8,9,10,11,12,13,14,15,16,17,18,19,20,21,22,23,24,25,26,27,28,29 & 0 \\ \hline
  			1 & 12:00 AM & 11:59 PM & 0,1,2,3,4,5,6,7,8,9,10,11,12,13,14,15,16,17,18,19,20,21,22,23,24,25,26,27,28,29 & 1 \\ \hline
  			2 & 12:00 AM & 11:59 PM & 0,1,2,3,4,5,6,7,8,9,10,11,12,13,14,15,16,17,18,19,20,21,22,23,24,25,26,27,28,29 & 2 \\ \hline
  			3 & 12:00 AM & 11:59 PM & 0,1,2,3,4,5,6,7,8,9,10,11,12,13,14,15,16,17,18,19,20,21,22,23,24,25,26,27,28,29 & 3 \\ \hline
  			4 & 12:00 AM & 11:59 PM & 0,1,2,3,4,5,6,7,8,9,10,11,12,13,14,15,16,17,18,19,20,21,22,23,24,25,26,27,28,29 & 4 \\ \hline
  			5 & 12:00 AM & 11:59 PM & 0,1,2,3,4,5,6,7,8,9,10,11,12,13,14,15,16,17,18,19,20,21,22,23,24,25,26,27,28,29 & 5 \\ \hline
  			6 & 12:00 AM & 11:59 PM & 0,1,2,3,4,5,6,7,8,9,10,11,12,13,14,15,16,17,18,19,20,21,22,23,24,25,26,27,28,29 & 6 \\ \hline
    	\end{tabular}
	\end{center}
	\end{table}
	\item \textbf{Expected Output:} Simulation outputs that it has completed each step and has written the results to the dashboard.
	\item \textbf{Actual Output:}\\
	*****SIMULATION COMPLETE*****\\
	*****WRITING OUTPUT*****\\
	*****SAVING OUTPUT*****\\
	*****BUILDING DASHBOARD*****\\
	*****DASHBOARD COMPLETE*****\\
\end{enumerate}

\subsection{Compatibility Test 1} %add info for cxfreeze
The simulation is run and compared to results computed on other operating systems to ensure it is functioning normally in the new environment. The stakeholders have indicated that they will only be using the software on versions of windows. The current implementation utilizes cxfreeze to create windows executable files. 
\begin{enumerate}[(i)]
	\item \textbf{Initial State:} Unknown Operating System with the simulation accessible on local storage
	\item \textbf{Input:} 'Termination Correctness 1' simulation parameters (details outline previously under 'Termination Correctness 1')
	\item \textbf{Expected Output:} Simulated results are consistent with previously generated values
	\item \textbf{Actual Output:} Currently, the simulation has been tested on a number of systems, including windows 7 and 8, and the executables produced by cxfreeze have performed perfectly. Final output from each simulation compatibility test has been: \\
	*****SIMULATION COMPLETE*****\\
	*****WRITING OUTPUT*****\\
	*****SAVING OUTPUT*****\\
	*****BUILDING DASHBOARD*****\\
	*****DASHBOARD COMPLETE*****\\
\end{enumerate}

\section{Nonfunctional Test Reports}
\subsection{Usability Test 1}
End user is provided with a set of instructions from the user manual on how to initialize and run the simulation. The success of this test is determined by how much external assistance the End User requires from the development team on their first use of the software. 
\begin{enumerate}[(i)]
	\item \textbf{Initial State:} Simulation prior to execution
	\item \textbf{Input:} End user with a copy of both the simulation software package and accompanying user manual.
	\item \textbf{Expected Output:} The end user successfully Initializes the Simulation
	\item \textbf{Actual Output:} The simulation is yet to be formally tested with stakeholders in regards to the scope of this test case. However, feedback has been gathered during project demos with a few key stakeholders, and the responses have been positive. This test will become critical during the very final stages of development, as the interface for the project is being tailored to the stakeholder's needs.
\end{enumerate}

\subsection{Performance Test 1}
The project stakeholders have placed a loose time limit on the duration of execution for a single simulation run of fifteen minutes. No correct simulation execution has taken longer than five minutes, with most between one and two minutes, but ensuring this general timing for the simulation across all applicable systems is the purpose of this test.
\begin{enumerate}[(i)]
	\item \textbf{Initial State:} Simulation prior to execution 
	\item \textbf{Input:} Any combination of inputs that the interface will accept and begin simulating
	\item \textbf{Expected Output:} The simulation completes execution taking between thirty seconds and five minutes, repeated thrice for consistency.
	\item \textbf{Actual Output:}
	\begin{table}
	\caption{Simulation Execution Times for Performance Test 1}
	\begin{center}
    	\begin{tabular}{| c | c |}
    		\hline
        	Simulation Run & Duration of Execution \\ \hline
  			1 & 2:45 \\ \hline
  			2 & 3:42 \\ \hline
  			3 & 3:17 \\ \hline
    	\end{tabular}
	\end{center}
	\end{table}
\end{enumerate}

\section{Summary}
%Changes made as a result of testing, talk about automated testing
In many of the above test cases, the user interface is used to specify simulation parameters, even though it may not be the focus of the particular test case. This is primarily for convenience, as the interface itself serves to automate and centralize some of the data required to begin a simulation. As such, many of our original automated test cases have been removed to account for the added capability of the interface.\\
\\
These test cases, whether performed as a formal testing suite or quick samples during implementation changes, have provided useful information for repairing or improving the simulation software. Many of these tests have produced expected results without fail, but they provide essential security by repeatedly checking the core features of the software, ensuring accuracy throughout development.   One test in particular, Task Pool Correctness 1, which overloads a single porter with many tasks to test the dispatcher's ability to store pending jobs, highlighted a major problem with how jobs were dispatched. Jobs from the beginning of a day weren't being handled by a porter until the very end, this led to an examination of the dispatcher module and the resolution of the issue. 
\section{Figures and Tables Appendix}
\begin{enumerate}[(a)]
	\item Figure 3.1: User Interface
	\item Table  3.1: Porter Shift Information in schedule.csv for Event List Correctness 1
	\item Table  3.2: Porter Shift Information in schedule.csv for State Change Correctness 1
	\item Table  3.3: Porter Shift Information in schedule.csv for Porter/Event Linkage Correctness 1
	\item Table  3.4: Porter Shift Information in schedule.csv for Task Pool Correctness 1
	\item Table  3.5: Porter Shift Information in schedule.csv for Termination Correctness 1
	\item Table  4.1: Simulation Execution Times for Performance Test 1
\end{enumerate}

%%% End document
\end{document}