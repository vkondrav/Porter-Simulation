%%%%%%%%%%%%%%%%%%%%%%%%%%%%%%%%%%%%%%%%%%%%%%%%%%%%%%%%%%%%%%%%%%%%%%
% LaTeX Example: Project Report
%
% Source: http://www.howtotex.com
%
% Feel free to distribute this example, but please keep the referral
% to howtotex.com
% Date: March 2011 
% 
%%%%%%%%%%%%%%%%%%%%%%%%%%%%%%%%%%%%%%%%%%%%%%%%%%%%%%%%%%%%%%%%%%%%%%
% How to use writeLaTeX: 
%
% You edit the source code here on the left, and the preview on the
% right shows you the result within a few seconds.
%
% Bookmark this page and share the URL with your co-authors. They can
% edit at the same time!
%
% You can upload figures, bibliographies, custom classes and
% styles using the files menu.
%
% If you're new to LaTeX, the wikibook is a great place to start:
% http://en.wikibooks.org/wiki/LaTeX
%
%%%%%%%%%%%%%%%%%%%%%%%%%%%%%%%%%%%%%%%%%%%%%%%%%%%%%%%%%%%%%%%%%%%%%%
% Edit the title below to update the display in My Documents
%\title{Project Report}
%
%%% Preamble
\documentclass[paper=letter, fontsize=10pt]{scrartcl}
\usepackage[T1]{fontenc}
\usepackage{fourier}

\usepackage[english]{babel}															% English language/hyphenation
\usepackage[protrusion=true,expansion=true]{microtype}	
\usepackage{amsmath,amsfonts,amsthm} % Math packages
\usepackage[pdftex]{graphicx}	
\usepackage{url}
\usepackage{enumerate}
\usepackage{lastpage}


%%% Custom sectioning
\usepackage{sectsty}
\allsectionsfont{\normalfont\scshape}


%%% Custom headers/footers (fancyhdr package)
\usepackage{fancyhdr}
\pagestyle{fancy}
\fancyhead[L]{}
\fancyhead[c]{Requirements Revision 0}
\fancyhead[R]{\today}											
\fancyfoot[L]{}											 
\fancyfoot[C]{}											
\fancyfoot[R]{\thepage\ of \pageref{LastPage}}		% Pagenumbering
\renewcommand{\headrulewidth}{0.4pt}				% Remove header underlines
\renewcommand{\footrulewidth}{0.4pt}				% Remove footer underlines
\setlength{\headheight}{13.6pt}


%%% Equation and float numbering
\numberwithin{equation}{section}		% Equationnumbering: section.eq#
\numberwithin{figure}{section}			% Figurenumbering: section.fig#
\numberwithin{table}{section}				% Tablenumbering: section.tab#


%%% Maketitle metadata
\newcommand{\horrule}[1]{\rule{\linewidth}{#1}} 	% Horizontal rule

%%% Begin document
\begin{document}

\begin{titlepage}

\newcommand{\HRule}{\rule{\linewidth}{0.5mm}} % Defines a new command for the horizontal lines, change thickness here

\begin{center}
 
%----------------------------------------------------------------------------------------
%	HEADING SECTIONS
%----------------------------------------------------------------------------------------

\textsc{\LARGE McMaster University}\\[1.5cm] % Name of your university/college
\textsc{\Large CAS 4ZP6 Capstone Project 2013/2014}\\[0.5cm] % Major heading such as course name
\textsc{\large Porter Simulation}\\[0.5cm] % Minor heading such as course title

%----------------------------------------------------------------------------------------
%	TITLE SECTION
%----------------------------------------------------------------------------------------

\HRule \\[0.4cm]
{ \huge \bfseries Requirements Documentation Revision 0}\\[0.4cm] % Title of your document
\HRule \\[1.5cm]
 
%----------------------------------------------------------------------------------------
%	AUTHOR SECTION
%----------------------------------------------------------------------------------------

\begin{minipage}{0.4\textwidth}
\begin{flushleft} \large
\emph{Authors:}\\
Vitaliy Kondratiev\\
Nathan Johrendt\\
Tyler Lyn\\
Mark Gammie
\end{flushleft}
\end{minipage}
~
\begin{minipage}{0.4\textwidth}
\begin{flushright} \large
\emph{Supervisor:} \\
Dr. Douglas Down % Supervisor's Name
\end{flushright}
\end{minipage}\\[4cm]

% If you don't want a supervisor, uncomment the two lines below and remove the section above
%\Large \emph{Author:}\\
%John \textsc{Smith}\\[3cm] % Your name

%----------------------------------------------------------------------------------------
%	DATE SECTION
%----------------------------------------------------------------------------------------

{\large \today}\\[3cm] % Date, change the \today to a set date if you want to be precise

%----------------------------------------------------------------------------------------
%	LOGO SECTION
%----------------------------------------------------------------------------------------

%\includegraphics{Logo}\\[1cm] % Include a department/university logo - this will require the graphicx package
 
%----------------------------------------------------------------------------------------

\vfill % Fill the rest of the page with whitespace
\end{center}
\end{titlepage}

\setcounter{tocdepth}{2}

\tableofcontents

\newpage

\section{Purpose of the Project}
\subsection{Background}
Hamilton Health Sciences are experiencing inefficiencies when synchronizing their porter services throughout each of their locations. Porter services, in this context, are defined as the movement of equipment such as beds, wheelchairs, other medical instruments and patient transfers from one location to another. Porters are a key piece of overall patient experience and satisfaction; the flow of day to day operations in a hospital depends on their efficiency. The problem HHS is facing is synchronization of porter services with the existing constraints.  Porters should be able to achieve greater efficiency by minimizing client (patient/doctor/nurse/technician/etc.) wait times and reducing the total time wasted in everyday operations. Hamilton Health Sciences are lacking the tools to solve this problem.
\subsection{Goals}
Our goal is to provide HHS with the tools to simulate their porter services so that they can test their own solutions, methods and make calculated decisions based on the results.

\section{The Stakeholders}
Hamilton Health Sciences (HHS) operational management team is the main stakeholder for this project. Names and position are: 
\subsection{The Client}
Hamilton Health Sciences (HHS) operational management team is the client for this project. Names and positions are:
\subsection{The Customer}
Hamilton Health Sciences (HHS) operational management team is also the customer for this project. Names and positions are:
\subsection{Other Stakeholders}
Patients and Hospital Staff are the secondary stakeholders for this project. Any benefits that arise from the successful completion of this project will affect these stakeholders.
\subsection{Hands on Users}
Operational Management Staff will be the primary hands on users of the finished product. Names and positions are:

\section{Mandated Constraints}
\subsection{Solution Constraints}
Currently there are no solution constraints as per discussion with the stakeholders.
\subsection{Schedule Constraints}
Simulation Software must be complete and requirements met by March 2014. (Current Academic Year)  


\section{Naming Conventions and Technology}
\subsection{Definitions of All Terms, Including Acronyms, Used by Stakeholders involved in the Project}
\begin{enumerate}[(a)]
	\item \textbf{HHS:} Hamilton Health Sciences
	\item \textbf{IVR:} Interactive Voice Request - phone system for requesting porter services
	\item \textbf{Porter:} Staff member responsible for movement of equipment such as beds, wheelchairs, other medical instruments and patient transfers from one location to another
	\begin{enumerate}[(i)]
		\item \textbf{Off-System Porter:} Porters that follow a strict scheduled and a predetermined set of activities
		\item \textbf{On-System Porter:} Porters that respond to ad-hoc and pre-booked requests	
	\end{enumerate}
	\item \textbf{Dispatching System:} An automated software system responsible for receiving and assigning requests to Porters
	\begin{enumerate}[(i)]
		\item \textbf{RDE:} Remote Dispatching E... - portal used to view the current state of all the porter jobs currently in the Dispatching System
	\end{enumerate}
	\item \textbf{Zoom Stretcher:} A powered stretched used for patient transportation
	\item \textbf{Pneumatic Tube:} A transportation system for lab samples and specimens
	\item \textbf{Priority of Requests:} Requests placed by Hospital Staff can be prioritized on a scale from 0 - 9 with 0 being the most urgent. Porters can place an Assist Call that has a higher priority than 0.
	\item \textbf{Event State:} A state of the porter service event as dictated by the Dispatching System
	\begin{enumerate}[(i)]
		\item \textbf{Pending:} Job has been placed in the system queue
		\item \textbf{Dispatched:} Job has been matched to an available porter
		\item \textbf{In-Progress:} Job is being executed by the porter
		\item \textbf{Complete:} Job has been completed
		\item \textbf{Dispatch Delay:} Porter states that he/she is delayed during a Dispatched event
		\item \textbf{In-Progress Delay:} Porter states that he/she is delayed during a In-Progress event
	\end{enumerate} 
	\item \textbf{Transaction Time:} the time from Event State (Pending) to Event State (Complete)
	\item \textbf{Proactive Page:} A porter pages the request location to inform the requester of his/her impending arrival      
	\item \textbf{Age of Request:} How long a job has been pending in the dispatch system
	\item \textbf{Break3:} A state a porter can enter where they are still logged into the system but do not receive any calls from the dispatch system
	\item \textbf{Zoned:} A term for a porter that only receives certain types of requests as specified by the dispatching system 
	\item \textbf{CSV file:} Comma Separated Values file, stores tabular data in plain text form. 
	\item \textbf{System Item Request:} requests that deal with specimens/blood/bed movement/etc \ldots
	\item \textbf{GUI:} Graphical User Interface - interaction with electronic devices through graphical icons and visual indicators
\end{enumerate}

\section{Relevant Facts and Assumptions}
\subsection{Relevant Facts}
\begin{enumerate}[(a)]
	\item HHS will provide the project team with available non critical data
	\item HHS currently uses a Dispatching System to rout its porters to desired locations
	\item On certain route segments, two porters are required to transport the patient 
	\item Lab/Specimen delivery use Off-System Porters on weekdays from 8:00 to 16:00
\end{enumerate}
\subsection{Business Rules}
\begin{enumerate}[(a)]
	\item Under their collective agreement a porter cannot be scheduled for less than an X amount of hours per week and X amount of hours per shift 
	\item Each request has six event states (Pending, Dispatch, In-Progress, Complete, Dispatch Delay, In-Progress Delay)
	\item Requests are prioritized on a 0 - 9 scale
	\item There are three types of transportation equipment (Stretcher, Zoom Stretcher, Wheelchair)
	\item There are two types of porters (On-System, Off-System)
	\item Every completed event has an associated transaction time
	\item Only Off-System porters can be scheduled for pre-booked requests
	\item Juravinski uses the Pneumatic Tube for 40\% of the lab deliveries and porters for 60\%
	\item Dispatch System determines the assignment of requests by using these parameters (Priority, Proximity, Pre-Scheduled Appointment Use, Age of Request) 
	\item Industry standard for patient transport transaction time is 30 minutes
	\item Porter service requests can be made using any hospital computer or phone (IVR)
	\item Porters can enter into an unscheduled break mode (Break3) where they do not receive any requests from the dispatch system
	\item Porters can be "zoned" into system item requests by the dispatching system  
\end{enumerate}	    
\subsection{Assumptions}
\begin{enumerate}[(a)]
	\item Every porter is equally capable of performing every task as every other porter
	\item All transporting equipment is in equal physical condition as other equipment of the same type
	\item Wheel Chair transport of patients is four to five minutes less in transaction time
	\item Some porters use proactive paging  
	\item The majority of service requests are made through hospital computers
	\item Porters do not abuse the "Break3" mode
\end{enumerate}

\section{Scope of the Work}
\subsection{Current Situation}
Hamilton Health Sciences are experiencing inefficiencies when synchronizing their porter services throughout each of their locations and are lacking the tools to solve this problem. The biggest problem comes from the lack of compliance and coordination of the many separate entities of the hospital body.
\subsection{Context of the Work}
The context of the work is to provide HHS with a simulation that models their porter services. The simulation is not to provide a solution but act as a tool to test new operational ideas.
\subsection{Business Use Case}
The simulation tool will be used by members of the operational management staff to model their process as per their variables and the simulation constraints.

\section{Scope of the Product}
\subsection{Product Boundary}
\begin{enumerate}[(a)]
	\item The simulation tool will not model the 100\% full hospital environment
	\item The simulation will focus on the On-System Porters
	\item Not all porter activities will be simulated. Simulation will concentrate on the 6 event states tracked by the dispatching system (Pending, Dispatch, In-Progress, Complete, Dispatch Delay, In-Progress Delay).
	\item The simulation tool will only model the Juravinski Hospital location  
\end{enumerate} 
\subsection{Product Use Cases}
\begin{enumerate}[(a)]
	\item Operational Manager has a new initiative they want to implement into everyday operation. He/She uses the simulation by changing the adjustable variables with his/her own values and executing it. He/She analyses the output of the simulation and determines if the new initiative should be implemented.
	\item Operational Manager has to determine how to modify the schedule for the porter service staff. He/She uses the simulation by changing the adjustable variables with his/her own values and executing it. He/She uses the output to design/refine the new schedule.
	\item Operational Manager wants to increase operational compliance of some particular policy. He/She uses the simulation by changing the adjustable variables related to a certain level of compliance with his/her own values and executing it. Once positive results have been verified he/she shows the results to all the parties involved in the compliance policy to effectively increase compliance.
	\item Operational Manager wants to experiment with theoretical scenarios. He/She uses the simulation by changing the adjustable variables with his/her own values and executing it. He/She analyses the output data and either creates a new initiative based on result or archives the data.
	     
\end{enumerate}

\section{Functional Requirements}
\subsection{Functional Requirements}
\begin{enumerate}[(a)]
	\item \textbf{Description:} Simulation must use a CSV file type as input
	\\ \textbf{Rationale:} Operational Management staff has indicated that using the CSV file type as the input is the preferred option
	\\ \textbf{Originator:} Operational Management staff
	\\ \textbf{Fit Criterion:} Simulation must accept the CSV File type without error 100\% of the time assuming the CSV file is without error
	\item \textbf{Description:} A series of simulation variables that affect the simulation output must be editable by the user
	\\ \textbf{Rationale:} Operational Management staff must be able to modify the simulation
	\\ \textbf{Fit Criterion:} Simulation must include at least 85\% of the following variables   
	\begin{enumerate}[(i)]
		\item Number of Porters
		\item Frequency of Events
		\item Number of Locations
		\item To be decided \ldots
	\end{enumerate}
	\item \textbf{Description:} Simulation Tool must be able to run a pre-designed model incorporating the given input variables and exit
	\\ \textbf{Rationale:} Operational Staff must be able to run the simulation
	\\ \textbf{Fit Criterion:} Simulation must be accurate within X range of accuracy 
	\item \textbf{Description} The output of the simulation must be a CSV file type
	\\ \textbf{Rationale:} It is best design to use the same file type for output as input
	\\ \textbf{Fit Criterion:} The output file must be have 0\% errors according to the CSV file type standards 
\end{enumerate}

\section{Look and Feel Requirements}
\subsection{Appearance Requirements}
Appearance Requirements are still in the discussion stage with the stakeholders.
\subsection{Style Requirements}
\begin{enumerate}[(a)]
	\item Software must contain elements of basic human/computer interface design as expected by a casual user of personal computers and popular software/operating systems. The exact details will be worked out after further discussion with the end users.
\end{enumerate}

\section{Usability and Humanity Requirements}
\subsection{Ease of Use Requirements}
\begin{enumerate}
	\item \textbf{Content:} Software must have a GUI
	\\	  \textbf{Motivation:} Users of this software are not assumed to be advanced computer users. Users are not expected to know how to use command line or similar interfaces. 
	\\	  \textbf{Fit Criterion:} All basic functions of the software are accessible through a GUI
	\\	  \textbf{Considerations:} This Ease of Use requirement considers all of the Product Use Cases
	\item \textbf{Content:} Software must be able to undo actions
	\\	  \textbf{Motivation:} As per the familiarity requirements the software must mimic some of the main functions of popular software
	\\	  \textbf{Fit Criterion:} All basic actions completed in the GUI will be able to be undone up to a certain point (to be decided)
	\\	  \textbf{Considerations:} This Ease of Use requirement considers all of the Product Use Cases
	\item \textbf{Content:} Software must be easy to navigate
	\\	  \textbf{Motivation:} Users should be able to easily move between different screens
	\\	  \textbf{Fit Criterion:} Each screen is linked to each other with an easily accessible interface feature
	\\	  \textbf{Considerations:} This Ease of Use requirement considers all of the Product Use Cases
	\item \textbf{Content:} User must clearly understand all the functions with minimal training
	\\	  \textbf{Motivation:} User should be able to pick up the functionality based on the context
	\\	  \textbf{Fit Criterion:} All elements of GUI should will be easy to understand under context of the usability
	\\	  \textbf{Consideration:} This Ease of Use requirement considers all of the Product Use Cases
\end{enumerate}
\subsection{Personalization and Internationalization Requirements}
Personalization of the final product will be discussed with stakeholders and end users once further development has occurred. Internationalization requirements are not applicable.
\subsection{Learning Requirements}
\begin{enumerate}[(a)]
	\item \textbf{Content:} Software must be easy to learn with some hands-on training and reading training documentation by a casual user of personal computers
	\\	  \textbf{Motivation:} Users are not required to have any knowledge of simulation software to operate the product
	\\	  \textbf{Fit Criterion:} Users will be able to use the software after a few training sessions of less than sixty minutes
	\\	  \textbf{Consideration:} This Learning requirement considers all of the Product Use Cases	
\subsection{Understandability and Politeness Requirements}
	\item \textbf{Content:} Users should be able to quickly understand how the software will benefit them in their business process
	\\	  \textbf{Motivation:} Users are not expected to understand aspects that do not directly relate to their purpose
	\\	  \textbf{Fit Criterion:} All of the simulated aspects will be related to the user's business problems unless the case considered is far out of the problem scope stated in these requirements
	\\	  \textbf{Consideration:} This Understandability requirement considers all of the Product Use Cases	
\end{enumerate}
\subsection{Accessibility Requirements}
Currently there have been no Accessibility Requirements from the stakeholders.

\section{Performance Requirements}
\subsection{Speed and Latency Requirements}
\begin{enumerate}[(a)]
	\item \textbf{Content:} Software must be able to complete the simulation as set up by the user within a reasonable time
	\\	  \textbf{Motivation:} As per request by the stakeholders
	\\	  \textbf{Fit Criterion:} A single simulation should not take more than a working day (12h) to complete
	\\	  \textbf{Considerations:} This speed requirement considers all of the Product Use Cases
\end{enumerate}
\subsection{Safety-Critical Requirements}
Currently there have been no Safety-Critical Requirements proposed by the stakeholders.
\subsection{Precision or Accuracy Requirements}
Precision and Accuracy requirements are subject to change during the course of the project. Simulation must be accurate within a certain range. To be decided at a later date. 
\subsection{Reliability and Availability Requirements}
	\begin{enumerate}[(a)]
		\item \textbf{Content:} Software must output relevant data to the user without error
		\\	  \textbf{Motivation:} Users should expect the output to be useful in their business process
		\\	  \textbf{Fit Criterion:} Output will be in correct format as per Functional Requirement \# 4
		\\	  \textbf{Considerations:} This reliability requirement considers all of the Product Use Cases 
		\item \textbf{Content:} Software must be available to the user at all times except when a simulation is running
		\\	  \textbf{Motivation:} Users should be able to access and use the software at any point in time
		\\	  \textbf{Fit Criterion:} Software is available to use 100\% of the time other than when the simulation is executing
		\\	  \textbf{Considerations:} This availability requirement considers all of the Product Use Cases 
	\end{enumerate}
\subsection{Robustness or Fault-Tolerance Requirements}
	\begin{enumerate}[(a)]
		\item \textbf{Content:} Software must not crash during the simulation process if the simulation is running within the scope of the project
		\\	  \textbf{Motivation:} Users should expect most simulations to complete without error
		\\	  \textbf{Fit Criterion:} The simulation should not fail 99\% of the time
		\\	  \textbf{Considerations:} This robustness requirement considers all of the Product Use Cases
	\end{enumerate}
\subsection{Capacity Requirements}
Capacity Requirements are still in the discussion stage with the stakeholders.
\subsection{Scalability or Extensibility Requirements}
Scalability and Extensibility Requirements are still in the discussion stage with the stakeholders.
\subsection{Longevity Requirements}
Longevity Requirements are still in the discussion stage with the stakeholders.

\section{Operational and Environmental Requirements}
\subsection{Expected Physical Environment}
Software on a computer station in a HHS employee's office.
\subsection{Requirements for Interfacing with Adjacent Systems}
Not Applicable.
\subsection{Productization Requirements}
Not Applicable.
\subsection{Release Requirements}
Final version of simulation software should be made available by March 2014.

\section{Maintainability and Support Requirements}
\subsection{Maintenance Requirements}
	\begin{enumerate}[(a)]
		\item The Project Team will provide maintenance to the software up to the projected project finish date (March 2014)
	\end{enumerate}
\subsection{Supportability Requirements}
	\begin{enumerate}[(a)]
		\item The Project Team will provide support for the software up to the projected project finish date (March 2014)
	\end{enumerate}
\subsection{Adaptability Requirements}
The simulation is being modeled after existing data provided by stakeholders, with the possibility of modifying the software later to accommodate updated base values. 

\section{Security Requirements}
\subsection{Access Requirements}
	\begin{enumerate}[(a)]
		\item Software will be accessible to any user who has access to the system the software resides on
	\end{enumerate}
\subsection{Integrity Requirements}
Not Applicable.
\subsection{Privacy Requirements}
Confidentiality waivers are required for project members to participate in on-site visits to HHS locations during simulation development.
\subsection{Audit Requirements}
Not Applicable.
\subsection{Immunity Requirements}
Not Applicable.

\section{Cultural Requirements}
\subsection{Cultural Requirements}
Not Applicable.

\section{Legal Requirements}
\subsection{Compliance Requirements}
Not Applicable.
\subsection{Standards Requirements}
Not Applicable.

\section{Off-the-Shelf Solutions}
\subsection{Ready-Made Products}
Visual8 produces visual process modeling simulations and have worked with HHS on past projects.
\subsection{Reusable Components}
Simulation model should be adaptable to other HHS locations utilizing on-system porter services.
\subsection{Products That Can Be Copied}
None found that are applicable and freely available to duplicate.

\section{New Problems}
\subsection{Effects on the Current Environment}
Only when the product simulates positive beneficial results consistently will stakeholders consider implementing modifications to the existing HHS environment.
\subsection{Effects on the Installed Systems}
The product should have no effect on installed systems or other software.
\subsection{Potential User Problems}
None applicable yet, to be determined.
\subsection{Limitations in the Anticipated Implementation Environment That May Inhibit the New Product}
The simulation software should only be executed on systems that meet the previously stated Performance Requirements.
\subsection{Follow-Up Problems}
None applicable yet, to be determined.

\section{Tasks}
\subsection{Project Planning}
\subsection{Planning of the Development Phases}

\section{Risks}

\section{Costs}
There are currently no financial costs associated with this project.

\section{User Documentation and Training}
\subsection{User Documentation Requirements}
\subsection{Training Requirements}

\section{Open Issues}
\begin{enumerate}
	\item Continue to narrow down complete variable list
	\item Clearly define which tasks on/off system porters complete.
	\item Establish specific precision and accuracy requirements.
\end{enumerate}

\section{Waiting Room}

\section{Ideas for Solutions}

%%% End document
\end{document}