%%%%%%%%%%%%%%%%%%%%%%%%%%%%%%%%%%%%%%%%%%%%%%%%%%%%%%%%%%%%%%%%%%%%%%
% LaTeX Example: Project Report
%
% Source: http://www.howtotex.com
%
% Feel free to distribute this example, but please keep the referral
% to howtotex.com
% Date: March 2011 
% 
%%%%%%%%%%%%%%%%%%%%%%%%%%%%%%%%%%%%%%%%%%%%%%%%%%%%%%%%%%%%%%%%%%%%%%
% How to use writeLaTeX: 
%
% You edit the source code here on the left, and the preview on the
% right shows you the result within a few seconds.
%
% Bookmark this page and share the URL with your co-authors. They can
% edit at the same time!
%
% You can upload figures, bibliographies, custom classes and
% styles using the files menu.
%
% If you're new to LaTeX, the wikibook is a great place to start:
% http://en.wikibooks.org/wiki/LaTeX
%
%%%%%%%%%%%%%%%%%%%%%%%%%%%%%%%%%%%%%%%%%%%%%%%%%%%%%%%%%%%%%%%%%%%%%%
% Edit the title below to update the display in My Documents
%\title{Project Report}
%
%%% Preamble
\documentclass[paper=letter, fontsize=10pt]{scrartcl}
\usepackage[T1]{fontenc}
\usepackage{fourier}

\usepackage[english]{babel}															% English language/hyphenation
\usepackage[protrusion=true,expansion=true]{microtype}	
\usepackage{amsmath,amsfonts,amsthm} % Math packages
\usepackage[pdftex]{graphicx}	
\usepackage{url}
\usepackage{enumerate}


%%% Custom sectioning
\usepackage{sectsty}
\allsectionsfont{\normalfont\scshape}


%%% Custom headers/footers (fancyhdr package)
\usepackage{fancyhdr}
\pagestyle{fancy}
\fancyhead[L]{}
\fancyhead[c]{Requirements Revision 0}
\fancyhead[R]{\today}											
\fancyfoot[L]{}											 
\fancyfoot[C]{}											
\fancyfoot[R]{\thepage\ of \pageref{}}		% Pagenumbering
\renewcommand{\headrulewidth}{0.4pt}				% Remove header underlines
\renewcommand{\footrulewidth}{0.4pt}				% Remove footer underlines
\setlength{\headheight}{13.6pt}


%%% Equation and float numbering
\numberwithin{equation}{section}		% Equationnumbering: section.eq#
\numberwithin{figure}{section}			% Figurenumbering: section.fig#
\numberwithin{table}{section}				% Tablenumbering: section.tab#


%%% Maketitle metadata
\newcommand{\horrule}[1]{\rule{\linewidth}{#1}} 	% Horizontal rule

%%% Begin document
\begin{document}

\begin{titlepage}

\newcommand{\HRule}{\rule{\linewidth}{0.5mm}} % Defines a new command for the horizontal lines, change thickness here

\begin{center}
 
%----------------------------------------------------------------------------------------
%	HEADING SECTIONS
%----------------------------------------------------------------------------------------

\textsc{\LARGE McMaster University}\\[1.5cm] % Name of your university/college
\textsc{\Large CAS 4ZP6 Capstone Project 2013/2014}\\[0.5cm] % Major heading such as course name
\textsc{\large Porter Simulation}\\[0.5cm] % Minor heading such as course title

%----------------------------------------------------------------------------------------
%	TITLE SECTION
%----------------------------------------------------------------------------------------

\HRule \\[0.4cm]
{ \huge \bfseries Requirements Documentation Revision 0}\\[0.4cm] % Title of your document
\HRule \\[1.5cm]
 
%----------------------------------------------------------------------------------------
%	AUTHOR SECTION
%----------------------------------------------------------------------------------------

\begin{minipage}{0.4\textwidth}
\begin{flushleft} \large
\emph{Authors:}\\
Vitaliy Kondratiev\\
Nathan Johrendt\\
Tyler Lyn\\
Mark Gammie
\end{flushleft}
\end{minipage}
~
\begin{minipage}{0.4\textwidth}
\begin{flushright} \large
\emph{Supervisor:} \\
Dr. Douglas Down % Supervisor's Name
\end{flushright}
\end{minipage}\\[4cm]

% If you don't want a supervisor, uncomment the two lines below and remove the section above
%\Large \emph{Author:}\\
%John \textsc{Smith}\\[3cm] % Your name

%----------------------------------------------------------------------------------------
%	DATE SECTION
%----------------------------------------------------------------------------------------

{\large \today}\\[3cm] % Date, change the \today to a set date if you want to be precise

%----------------------------------------------------------------------------------------
%	LOGO SECTION
%----------------------------------------------------------------------------------------

%\includegraphics{Logo}\\[1cm] % Include a department/university logo - this will require the graphicx package
 
%----------------------------------------------------------------------------------------

\vfill % Fill the rest of the page with whitespace
\end{center}
\end{titlepage}

\setcounter{tocdepth}{2}

\tableofcontents

\newpage

\section{Purpose of the Project}

\subsection{Background}

\subsection{Goals}

\section{The Stakeholders}
Hamilton Health Sciences (HHS) operational management team is the main stakeholder for this project. Names and position are:
Patients and Hospital Staff are the secondary stakeholders for this project. Any benefits that arise from the successful completion of this project will affect these stakeholders. 
\subsection{The Client}
Hamilton Health Sciences (HHS) operational management team is the client for this project. Names and positions are:
\subsection{The Customer}
Hamilton Health Sciences (HHS) operational management team is also the customer for this project. Names and positions are:
\subsection{Other Stakeholders}
Not Applicable
\subsection{Hands on Users}
Operational Management Staff will be the primary hands on users of the finished product. Names and positions are:

\section{Mandated Constraints}

\subsection{Solution Constraints}
\begin{enumerate}[(a)]
	\item 
\end{enumerate}
\subsection{Schedule Constraints}
\begin{enumerate}[(a)]
	\item Simulation Software must be complete and requirements met by March 2014 (Current Academic Year)  
\end{enumerate}

\section{Naming Conventions and Technology}
\subsection{Definitions of All Terms, Including Acronyms, Used by Stakeholders involved in the Project}
\begin{enumerate}[(a)]
	\item \textbf{HHS:} Hamilton Health Sciences
	\item \textbf{IVR:} Interactive Voice Request - phone system for requesting portering services
	\item \textbf{Porter:} Staff member responsible for movement of equipment such as beds, wheelchairs, other medical instruments and patient transfers from one location to another
	\begin{enumerate}[(i)]
		\item \textbf{Off-System Porter:} Porters that follow a strict scheduled and a predetermined set of activities
		\item \textbf{On-System Porter:} Porters that respond to ad-hoc and pre-booked requests	
	\end{enumerate}
	\item \textbf{Dispatching System:} An automated software system responsible for receiving and assigning requests to Porters
	\begin{enumerate}[(i)]
		\item \textbf{RDE} Remote Dispatching E... - portal used to view the current state of all the porter jobs currently in the Dispatching System
	\end{enumerate}
	\item \textbf{Zoom Stretcher:} A powered stretched used for patient transportation
	\item \textbf{Pneumatic Tube:} A transportation system for lab samples and specimens
	\item \textbf{Priority of Requests:} Requests placed by Hospital Staff can be prioritized on a scale from 0 - 9 with 0 being the most urgent. Porters can place an Assist Call that has a higher priority than 0.
	\item \textbf{Event State:} A state of the porter service event as dictated by the Dispatching System
	\begin{enumerate}[(i)]
		\item \textbf{Pending:} Job has been placed in the system queue
		\item \textbf{Dispatched:} Job has been matched to an available porter
		\item \textbf{In-Progress:} Job is being executed by the porter
		\item \textbf{Complete:} Job has been completed
		\item \textbf{Dispatch Delay:} Porter states that he/she is delayed during a Dispatched event
		\item \textbf{In-Progress Delay:} Porter states that he/she is delayed during a In-Progress event
	\end{enumerate} 
	\item \textbf{Transaction Time:} the time from Event State: Pending to Event State: Complete
	\item \textbf{Proactive Page:} A porter pages the request location to inform the requester of his/her impending arrival      
	\item \textbf{Age of Request:} How long a job has been pending in the dispatch system
	\item \textbf{Break3:} A state a porter can enter where they are still logged into the system but do not receive any calls from the dispatch system
	\item \textbf{Zoned:} A term for a porter that only receives certain types of requests as specified by the dispatching system 
	\item \textbf{CSV file:} Comma Separated Values file, stores tabular data in plain text form. 
	\item \textbf{System Item Request:} requests that deal with specimens/blood/bed movement/etc \ldots
\end{enumerate}

\section{Relevant Facts and Assumptions}
\subsection{Relevant Facts}
\begin{enumerate}[(a)]
	\item HHS will provide the project team with available non critical data
	\item HHS currently uses a Dispatching System to rout its porters to desired locations
	\item On certain route segments, 2 porters are required to transport the patient 
	\item Lab/Specimen delivery use Off-System Porters on weekdays from 8:00 to 16:00
\end{enumerate}
\subsection{Business Rules}
\begin{enumerate}[(a)]
	\item Under their collective agreement a porter cannot be scheduled for less than an X amount of hours per week and X amount of hours per shift 
	\item Each request has 6 event states (Pending, Dispatch, In-Progress, Complete, Dispatch Delay, In-Progress Delay)
	\item Requests are prioritized on a 0 - 9 scale
	\item There are 3 types of transportation equipment (Stretcher, Zoom Stretcher, Wheelchair)
	\item There are two types of porters (On-System, Off-System)
	\item Every completed event has an associated transaction time
	\item Only Off-System porters can be scheduled for pre-booked requests
	\item Juravinski uses the Pneumatic Tube for 40\% of the lab deliveries and porters for 60 \%
	\item Dispatch System determines the assignment of requests by using these parameters (Priority, Proximity, Pre-Scheduled Appointment Use, Age of Request) 
	\item Industry standard for patient transport transaction time is 30 minutes
	\item Porter service requests can be made using any hospital computer or phone (IVR)
	\item Porters can enter into an unscheduled break mode (Break3) where they do not receive any requests from the dispatch system
	\item Porters can be "zoned" into system item requests by the dispatching system  
\end{enumerate}	    
\subsection{Assumptions}
\begin{enumerate}[(a)]
	\item Every porter is equally capable of performing every task as every other porter
	\item All transporting equipment is in equal physical condition as other equipment of the same type
	\item Wheel Chair transport of patients is 4-5min less in transaction time
	\item Some porters use proactive paging  
	\item The majority of service requests are made through hospital computers
	\item Porters do not abuse the "Break3" mode
\end{enumerate}

\section{Scope of the Work}
\subsection{Current Situation}
Hamilton Health Sciences are experiencing inefficiencies when synchronizing their porter services throughout each of their locations and are lacking the tools to solve this problem. The biggest problem comes from the lack of compliance and coordination of the many separate entities of the hospital body.
\subsection{Context of the Work}
The context of the work is to provide HHS with a simulation that models their porter services. The simulation is not to provide a solution but act as a tool to test new operational ideas.
\subsection{Business Use Case}
The simulation tool will be used by members of the operational management staff to model their process as per their variables and the simulation constraints.

\section{Scope of the Product}
\subsection{Product Boundary}
\begin{enumerate}[(a)]
	\item The simulation tool will not model the 100\% full hospital environment
	\item The simulation will focus on the On-System Porters
	\item Not all porter activities will be simulated. Simulation will concentrate on the 4 event states tracked by the dispatching system (Pending, Dispatch, In-Progress, Complete)
	\item The simulation tool will only model the Juravinski Hospital location  
\end{enumerate} 
\subsection{Product Use Cases}
\begin{enumerate}[(a)]
	\item Operational Manager has a new initiative they want to implement into everyday operation. He/She uses the simulation by changing the adjustable variables with his/her own values and executing it. He/She analyses the output of the simulation and determines if the new initiative should be implemented.
	\item Operational Manager has to determine how to modify the schedule for the porter service staff. He/She uses the simulation by changing the adjustable variables with his/her own values and executing it. He/She uses the output to design/refine the new schedule.
	\item Operational Manager wants to increase operational compliance of some particular policy. He/She uses the simulation by changing the adjustable variables related to a certain level of compliance with his/her own values and executing it. Once positive results have been verified he/she shows the results to all the parties involved in the compliance policy to effectively increase compliance.
	\item Operational Manager wants to experiment with theoretical scenarios. He/She uses the simulation by changing the adjustable variables with his/her own values and executing it. He/She analyses the output data and either creates a new initiative based on result or archives the data.
	     
\end{enumerate}

\section{Functional Requirements}
\subsection{Functional Requirements}
\begin{enumerate}[(a)]
	\item \textbf{Description:} Simulation must use a CSV file type as input
	\\ \textbf{Rationale:} Operational Management staff has indicated that using the CSV file type as the input is the preferred option
	\\ \textbf{Originator:} Operational Management staff
	\\ \textbf{Fit Criterion:} Simulation must accept the CSV File type without error 100\% of the time assuming the CSV file is without error
	\item \textbf{Description:} A series of simulation variables that affect the simulation output must be editable by the user
	\\ \textbf{Rationale:} Operational Management staff must be able to modify the simulation
	\\ \textbf{Fit Criterion:} Simulation must include at least 85\% of the following variables   
	\begin{enumerate}[(i)]
		\item Number of Porters
		\item Frequency of Events
		\item Number of Locations
	\end{enumerate}
	\item \textbf{Description:} Simulation Tool must be able to run a pre-designed model incorporating the given input variables and exit
	\\ \textbf{Rationale:} Operational Staff must be able to run the simulation
	\\ \textbf{Fit Criterion:} Simulation must be accurate within X range of accuracy 
	\item \textbf{Description} The output of the simulation must be a CSV file type
	\\ \textbf{Rationale:} It is best design to use the same file type for output as input
	\\ \textbf{Fit Criterion:} The output file must be have 0\% errors according to the CSV file type standards 
\end{enumerate}

\section{Look and Feel Requirements}
\subsection{Appearance Requirements}
Layout of input and output variables in the CSV files should be organized logically with descriptive titles. 
\subsection{Style Requirements}
No specific styles required.

\section{Usability and Humanity Requirements}
\subsection{Personalization and Internationalization Requirements}
None Applicable.
\subsection{Learning Requirements}
Users are familiar with CSV files and other simulation software. Learning required should be negligible.
\subsection{Understandability and Politeness Requirements}
Concepts and terms used in the simulation software and documentation are clearly understood by the intended users.
\subsection{Accessibility Requirements}
None Applicable.

\section{Performance Requirements}
\subsection{Speed and Latency Requirements}
Execution time of a single simulation set should be between a few seconds and 10 minutes.
\subsection{Safety-Critical Requirements}
None Applicable.
\subsection{Precision or Accuracy Requirements}

\subsection{Reliability and Availability Requirements}
Final revision of simulation software should be error-free and terminate consistently given proper input values.
\subsection{Robustness or Fault-Tolerance Requirements}
Simulation software should execute correctly when provided proper input and inform the user when incorrect values are given.
\subsection{Capacity Requirements}
None Applicable.
\subsection{Scalability or Extensibility Requirements}
None within the scope of this project.
\subsection{Longevity Requirements}
None within the scope of this project.

\section{Operational and Environmental Requirements}
\subsection{Expected Physical Environment}
A computer station in a HHS employee's office.
\subsection{Requirements for Interfacing with Adjacent Systems}
None Applicable.
\subsection{Productization Requirements}
None Applicable.
\subsection{Release Requirements}
Final version of simulation software should be made available by March 2014.

\section{Maintainability and Support Requirements}
\subsection{Maintenance Requirements}
None Applicable.
\subsection{Supportability Requirements}
Software to view and edit CSV files, such as Excell, is required to interact with the simulation.
\subsection{Adaptability Requirements}
The simulation is being modelled after existing data, with the possibility of modifying the software later to accommodate updated base values. 

\section{Security Requirements}
\subsection{Access Requirements}
None Applicable.
\subsection{Integrity Requirements}
None Applicable.
\subsection{Privacy Requirements}
Confidentiality waivers are required for project members to participate in on-site visits to HHS locations during simulation development.
\subsection{Audit Requirements}
None Applicable.
\subsection{Immunity Requirements}
None Applicable.

\section{Cultural Requirements}
\subsection{Cultural Requirements}
None Applicable.
\subsection{Legal Requirements}
None Applicable.
\subsection{Compliance Requirements}
None Applicable.
\subsection{Standards Requirements}
None Applicable.

\section{Open Issues}

\section{Off-the-Shelf Solutions}
\subsection{Ready-Made Products}
\subsection{Reusable Components}
Simulation model should be adaptable to other HHS locations utilizing on-system porter services.
\subsection{Products That Can Be Copied}

\section{New Problems}
\subsection{Effects on the Current Environment}
\subsection{Effects on the Installed Systems}
\subsection{Potential User Problems}
\subsection{Limitations in the Anticipated Implementation Environment That May Inhibit the New Product}
\subsection{Follow-Up Problems}

\section{Tasks}
\subsection{Project Planning}
\subsection{Planning of the Development Phases}

\section{Migration to the New Product}
\subsection{Requirements for Migration to the New Product}
\subsection{Data that Has To Be Modified or Translated for the New System}

\section{Risks}

\section{Costs}

\section{User Documentation and Training}
\subsection{User Documentation Requirements}
\subsection{Training Requirements}

\section{Waiting Room}

\section{Ideas for Solutions}

%%% End document
\end{document}
