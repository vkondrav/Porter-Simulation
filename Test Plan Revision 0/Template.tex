%%%%%%%%%%%%%%%%%%%%%%%%%%%%%%%%%%%%%%%%%%%%%%%%%%%%%%%%%%%%%%%%%%%%%%
% LaTeX Example: Project Report
%
% Source: http://www.howtotex.com
%
% Feel free to distribute this example, but please keep the referral
% to howtotex.com
% Date: March 2011 
% 
%%%%%%%%%%%%%%%%%%%%%%%%%%%%%%%%%%%%%%%%%%%%%%%%%%%%%%%%%%%%%%%%%%%%%%
% How to use writeLaTeX: 
%
% You edit the source code here on the left, and the preview on the
% right shows you the result within a few seconds.
%
% Bookmark this page and share the URL with your co-authors. They can
% edit at the same time!
%
% You can upload figures, bibliographies, custom classes and
% styles using the files menu.
%
% If you're new to LaTeX, the wikibook is a great place to start:
% http://en.wikibooks.org/wiki/LaTeX
%
%%%%%%%%%%%%%%%%%%%%%%%%%%%%%%%%%%%%%%%%%%%%%%%%%%%%%%%%%%%%%%%%%%%%%%
% Edit the title below to update the display in My Documents
%\title{Project Report}
%
%%% Preamble
\documentclass[paper=letter, fontsize=10pt]{scrartcl}
\usepackage[T1]{fontenc}
\usepackage{fourier}

\usepackage[english]{babel}															% English language/hyphenation
\usepackage[protrusion=true,expansion=true]{microtype}	
\usepackage{amsmath,amsfonts,amsthm} % Math packages
\usepackage[pdftex]{graphicx}	
\usepackage{url}
\usepackage{enumerate}
\usepackage{lastpage}


%%% Custom sectioning
\usepackage{sectsty}
\allsectionsfont{\normalfont\scshape}


%%% Custom headers/footers (fancyhdr package)
\usepackage{fancyhdr}
\pagestyle{fancy}
\fancyhead[L]{}
\fancyhead[c]{Requirements Revision 0}
\fancyhead[R]{\today}											
\fancyfoot[L]{}											 
\fancyfoot[C]{}											
\fancyfoot[R]{\thepage\ of \pageref{LastPage}}		% Pagenumbering
\renewcommand{\headrulewidth}{0.4pt}				% Remove header underlines
\renewcommand{\footrulewidth}{0.4pt}				% Remove footer underlines
\setlength{\headheight}{13.6pt}


%%% Equation and float numbering
\numberwithin{equation}{section}		% Equationnumbering: section.eq#
\numberwithin{figure}{section}			% Figurenumbering: section.fig#
\numberwithin{table}{section}				% Tablenumbering: section.tab#


%%% Maketitle metadata
\newcommand{\horrule}[1]{\rule{\linewidth}{#1}} 	% Horizontal rule
\newcommand{\ts}{\textsuperscript}

%%% Begin document
\begin{document}

\begin{titlepage}

\newcommand{\HRule}{\rule{\linewidth}{0.5mm}} % Defines a new command for the horizontal lines, change thickness here

\begin{center}
 
%----------------------------------------------------------------------------------------
%	HEADING SECTIONS
%----------------------------------------------------------------------------------------

\textsc{\LARGE McMaster University}\\[1.5cm] % Name of your university/college
\textsc{\Large CAS 4ZP6 Capstone Project 2013/2014}\\[0.5cm] % Major heading such as course name
\textsc{\large Porter Simulation}\\[0.5cm] % Minor heading such as course title

%----------------------------------------------------------------------------------------
%	TITLE SECTION
%----------------------------------------------------------------------------------------

\HRule \\[0.4cm]
{ \huge \bfseries Test Plan Revision 0}\\[0.4cm] % Title of your document
\HRule \\[1.5cm]
 
%----------------------------------------------------------------------------------------
%	AUTHOR SECTION
%----------------------------------------------------------------------------------------

\begin{minipage}{0.4\textwidth}
\begin{flushleft} \large
\emph{Authors:}\\
Vitaliy Kondratiev\\
Nathan Johrendt\\
Tyler Lyn\\
Mark Gammie
\end{flushleft}
\end{minipage}
~
\begin{minipage}{0.4\textwidth}
\begin{flushright} \large
\emph{Supervisor:} \\
Dr. Douglas Down % Supervisor's Name
\end{flushright}
\end{minipage}\\[4cm]

% If you don't want a supervisor, uncomment the two lines below and remove the section above
%\Large \emph{Author:}\\
%John \textsc{Smith}\\[3cm] % Your name

%----------------------------------------------------------------------------------------
%	DATE SECTION
%----------------------------------------------------------------------------------------

{\large \today}\\[3cm] % Date, change the \today to a set date if you want to be precise

%----------------------------------------------------------------------------------------
%	LOGO SECTION
%----------------------------------------------------------------------------------------

%\includegraphics{Logo}\\[1cm] % Include a department/university logo - this will require the graphicx package
 
%----------------------------------------------------------------------------------------
%Template taken from: http://www.softwaretestinghelp.com/test-plan-sample-softwaretesting-and-quality-assurance-templates/

\vfill % Fill the rest of the page with whitespace
\end{center}
\end{titlepage}

\setcounter{tocdepth}{2}

\tableofcontents

\newpage
\section{Introduction}

\section{Test Factors and Rationales}
\subsection{Reliability}
\subsection{Ease of Use}
\subsection{Portability}
\subsection{Correctness}

\section{Specific System Tests}
\subsection{Test 1}
\begin{enumerate}[a]
	\item \textbf{Test Factor:}  
	\item \textbf{Life Cycle Phase:}
	\item \textbf{Type:}
	\item \textbf{Dynamic/Static:}
	\item \textbf{Manual/Automated:}
	\item \textbf{Technique:}
		\begin{enumerate}[i]
			\item \textbf{Initial State:} Simulation event list is generated			
			\item \textbf{Input:} Event List
			\item \textbf{Description:} Event should be popped from the list then executed  
			\item \textbf{Expected Output:} Event list with one less element
		\end{enumerate}
\end{enumerate}

\subsection{Test 2}
\begin{enumerate}[a]
	\item \textbf{Test Factor:}  
	\item \textbf{Life Cycle Phase:}
	\item \textbf{Type:}
	\item \textbf{Dynamic/Static:}
	\item \textbf{Manual/Automated:}
	\item \textbf{Technique:}
		\begin{enumerate}[i]
			\item \textbf{Initial State:} Existing event list
			\item \textbf{Input:} Single Event (Any)
			\item \textbf{Description:} The execution of an event triggers a state change in the system
			\item \textbf{Expected Output:} State Change (Any)
		\end{enumerate}
\end{enumerate}

\subsection{Test 3}
\begin{enumerate}[a]
	\item \textbf{Test Factor:}  
	\item \textbf{Life Cycle Phase:}
	\item \textbf{Type:}
	\item \textbf{Dynamic/Static:}
	\item \textbf{Manual/Automated:}
	\item \textbf{Technique:}
		\begin{enumerate}[i]
			\item \textbf{Initial State:} Porters and Events exist
			\item \textbf{Input:} Porter (Pending State), Event
			\item \textbf{Description:} The available porter is linked to the event
			\item \textbf{Expected Output:} Porter (Dispatched State)
		\end{enumerate}
\end{enumerate}

\subsection{Test 4}
\begin{enumerate}[a]
	\item \textbf{Test Factor:}  
	\item \textbf{Life Cycle Phase:}
	\item \textbf{Type:}
	\item \textbf{Dynamic/Static:}
	\item \textbf{Manual/Automated:}
	\item \textbf{Technique:}
		\begin{enumerate}[i]
			\item \textbf{Initial State:} No porters with "Available State" 
			\item \textbf{Input:} Event
			\item \textbf{Description:} An event that cannot be assigned to any porter is put into a Task Pool
			\item \textbf{Expected Output:} Task Pool with Event
		\end{enumerate}
\end{enumerate}

\subsection{Test 5}
\begin{enumerate}[a]
	\item \textbf{Test Factor:}  
	\item \textbf{Life Cycle Phase:}
	\item \textbf{Type:}
	\item \textbf{Dynamic/Static:}
	\item \textbf{Manual/Automated:}
	\item \textbf{Technique:}
		\begin{enumerate}[i]
			\item \textbf{Initial State:} Simulation Finish  
			\item \textbf{Input:} State
			\item \textbf{Description:} After reaching end condition simulation ends
			\item \textbf{Expected Output:} Output file
		\end{enumerate}
\end{enumerate}

\subsection{Test 6}
\begin{enumerate}[a]
	\item \textbf{Test Factor:}  
	\item \textbf{Life Cycle Phase:}
	\item \textbf{Type:}
	\item \textbf{Dynamic/Static:}
	\item \textbf{Manual/Automated:}
	\item \textbf{Technique:}
		\begin{enumerate}[i]
			\item \textbf{Initial State:} Uninitialized Simulation State  
			\item \textbf{Input:} Simulation Input File
			\item \textbf{Description:} Input file will change the values of the simulation
			\item \textbf{Expected Output:} Initialized Simulation
		\end{enumerate}
\end{enumerate}

\subsection{Test 7}
\begin{enumerate}[a]
	\item \textbf{Test Factor:}  
	\item \textbf{Life Cycle Phase:}
	\item \textbf{Type:}
	\item \textbf{Dynamic/Static:}
	\item \textbf{Manual/Automated:}
	\item \textbf{Technique:}
		\begin{enumerate}[i]
			\item \textbf{Initial State:} Uninitialized Simulation State  
			\item \textbf{Input:} Simulation Input File
			\item \textbf{Description:} Input file will change the values of the simulation
			\item \textbf{Expected Output:} Initialized Simulation
		\end{enumerate}
\end{enumerate}

\section{Executive Summary}

\section{Schedule}

%%% End document
\end{document}