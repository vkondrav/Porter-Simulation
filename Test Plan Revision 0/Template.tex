%%%%%%%%%%%%%%%%%%%%%%%%%%%%%%%%%%%%%%%%%%%%%%%%%%%%%%%%%%%%%%%%%%%%%%
% LaTeX Example: Project Report
%
% Source: http://www.howtotex.com
%
% Feel free to distribute this example, but please keep the referral
% to howtotex.com
% Date: March 2011 
% 
%%%%%%%%%%%%%%%%%%%%%%%%%%%%%%%%%%%%%%%%%%%%%%%%%%%%%%%%%%%%%%%%%%%%%%
% How to use writeLaTeX: 
%
% You edit the source code here on the left, and the preview on the
% right shows you the result within a few seconds.
%
% Bookmark this page and share the URL with your co-authors. They can
% edit at the same time!
%
% You can upload figures, bibliographies, custom classes and
% styles using the files menu.
%
% If you're new to LaTeX, the wikibook is a great place to start:
% http://en.wikibooks.org/wiki/LaTeX
%
%%%%%%%%%%%%%%%%%%%%%%%%%%%%%%%%%%%%%%%%%%%%%%%%%%%%%%%%%%%%%%%%%%%%%%
% Edit the title below to update the display in My Documents
%\title{Project Report}
%
%%% Preamble
\documentclass[paper=letter, fontsize=10pt]{scrartcl}
\usepackage[body={7in,7.5in},top=1in, bottom=1in]{geometry}
\usepackage[T1]{fontenc}
\usepackage{fourier}

\usepackage[english]{babel}															% English language/hyphenation
\usepackage[protrusion=true,expansion=true]{microtype}	
\usepackage{amsmath,amsfonts,amsthm} % Math packages
\usepackage[pdftex]{graphicx}	
\usepackage{url}
\usepackage{enumerate}
\usepackage{lastpage}


%%% Custom sectioning
\usepackage{sectsty}
\allsectionsfont{\normalfont\scshape}


%%% Custom headers/footers (fancyhdr package)
\usepackage{fancyhdr}
\pagestyle{fancy}
\fancyhead[L]{}
\fancyhead[c]{Requirements Revision 0}
\fancyhead[R]{\today}											
\fancyfoot[L]{}											 
\fancyfoot[C]{}											
\fancyfoot[R]{\thepage\ of \pageref{LastPage}}		% Pagenumbering
\renewcommand{\headrulewidth}{0.4pt}				% Remove header underlines
\renewcommand{\footrulewidth}{0.4pt}				% Remove footer underlines
\setlength{\headheight}{13.6pt}


%%% Equation and float numbering
\numberwithin{equation}{section}		% Equationnumbering: section.eq#
\numberwithin{figure}{section}			% Figurenumbering: section.fig#
\numberwithin{table}{section}				% Tablenumbering: section.tab#


%%% Maketitle metadata
\newcommand{\horrule}[1]{\rule{\linewidth}{#1}} 	% Horizontal rule
\newcommand{\ts}{\textsuperscript}

%%% Begin document
\begin{document}

\begin{titlepage}

\newcommand{\HRule}{\rule{\linewidth}{0.5mm}} % Defines a new command for the horizontal lines, change thickness here
\newcommand{\authors}{\shortstack{Vitaliy Kondratiev,\\Nathan Johrendt,\\Tyler Lyn,\\Mark Gammie}}

\begin{center}
 
%----------------------------------------------------------------------------------------
%	HEADING SECTIONS
%----------------------------------------------------------------------------------------

\textsc{\LARGE McMaster University}\\[1.5cm] % Name of your university/college
\textsc{\Large CAS 4ZP6}\\[0.5cm]
\textsc{\Large Team 9} \\[0.5cm]
\textsc{\Large Capstone Project 2013/2014}\\[0.5cm] % Major heading such as course name
\textsc{\large Porter Simulation}\\[0.5cm] % Minor heading such as course title

%----------------------------------------------------------------------------------------
%	TITLE SECTION
%----------------------------------------------------------------------------------------

\HRule \\[0.4cm]
{ \huge \bfseries Test Plan Revision 0}\\[0.4cm] % Title of your document
\HRule \\[1.5cm]
 
%----------------------------------------------------------------------------------------
%	AUTHOR SECTION
%----------------------------------------------------------------------------------------

\begin{minipage}{0.4\textwidth}
\begin{flushleft} \large
\emph{Authors:}\\
Vitaliy Kondratiev\\
Nathan Johrendt\\
Tyler Lyn\\
Mark Gammie
\end{flushleft}
\end{minipage}
~
\begin{minipage}{0.4\textwidth}
\begin{flushright} \large
\emph{Supervisor:} \\
Dr. Douglas Down % Supervisor's Name
\end{flushright}
\end{minipage}\\[4cm]

% If you don't want a supervisor, uncomment the two lines below and remove the section above
%\Large \emph{Author:}\\
%John \textsc{Smith}\\[3cm] % Your name

%----------------------------------------------------------------------------------------
%	DATE SECTION
%----------------------------------------------------------------------------------------

{\large \today}\\[3cm] % Date, change the \today to a set date if you want to be precise

%----------------------------------------------------------------------------------------
%	LOGO SECTION
%----------------------------------------------------------------------------------------

%\includegraphics{Logo}\\[1cm] % Include a department/university logo - this will require the graphicx package
 
%----------------------------------------------------------------------------------------
%Template taken from: http://www.softwaretestinghelp.com/test-plan-sample-softwaretesting-and-quality-assurance-templates/

\vfill % Fill the rest of the page with whitespace
\end{center}
\end{titlepage}

\setcounter{tocdepth}{2}

\tableofcontents

\newpage

\section{Revision History}
\begin{center}
    \begin{tabular}{| c | l | l | l |}
    \hline
    Revision \# & Author & Date & Comment \\ \hline
  	1 & \shortstack{\\Vitaliy Kondratiev,\\Nathan Johrendt,\\Tyler Lyn,\\Mark Gammie} & October 28 & Adding Test Plan Revision 0 \\ \hline
  	2 & \shortstack{\\Vitaliy Kondratiev,\\Nathan Johrendt,\\Tyler Lyn,\\Mark Gammie} & October 29 & Test Plan Updates \\ \hline
  	3 & \shortstack{\\Vitaliy Kondratiev,\\Nathan Johrendt,\\Tyler Lyn,\\Mark Gammie} & October 29 & Test Plan Update \\ \hline
  	4 & \shortstack{\\Vitaliy Kondratiev,\\Nathan Johrendt,\\Tyler Lyn,\\Mark Gammie} & October 30 & Final Update for Revision 0 Test Plan\\
    \hline
    \end{tabular}
\end{center}

\section{Executive Summary}
\subsection{Introduction}
This document is designed to outline testing methods and techniques that are to be used during and after development
of the Porter Simulation. Below listed in detail are the main test factors and the rationale for choosing them. 
Following the main test factors is a comprehensive list of specific system tests including information about the Test's factor, life cycle phase, type, whether it is static or dynamic, manual or automated, and the specific techniques used to
conduct the test.
\subsection{Automation Tools}
\begin{enumerate}
	\item \textbf{Mock:} Mock is a Python library used for easy isolation of Python functions for testing and assertions
	\item \textbf{Unittest:} Unittest is a Python framework for constructing automated tests
	\item \textbf{Excel:} Excel will be used for analysing statistical information and verifying its correctness
	\item \textbf{Virtual Box:} Virtual Box will be used when testing the portability of our software between different operating systems
\end{enumerate}


\section{Test Factors and Rationales}
\subsection{Reliability}
\textbf{Rationale:} Since the simulation software is to be used by a non-technical staff, consistent execution and termination of the program is required. Individual simulations could run for long periods of time without requiring user interaction, and are expected to terminate and store results without user supervision.
\subsection{Ease of Use}
\textbf{Rationale:} End users are non-technical so any interaction with the program, whether input or output, should contain a minimal amount of technical information.  
\subsection{Portability} 
\textbf{Rationale:} Control of the systems that the Simulation will run on is left to the end users, so the implementation will be designed to function on a wide variety of industry-popular operating systems.
\subsection{Correctness}
\textbf{Rationale:} Since the Simulation will be reporting statistical data, the Simulation must be correct in managing and manipulating that data.  

\section{Specific System Tests}
\subsection{Event List Correctness 1}
\begin{enumerate}[]
	\item \textbf{Test Factor:} Correctness
	\item \textbf{Life Cycle Phase:} Throughout Development
	\item \textbf{Type:} Functional
	\item \textbf{Dynamic/Static:} Dynamic
	\item \textbf{Manual/Automated:} Manual
	\item \textbf{Technique:} 
		\begin{enumerate}[(i)]
			\item \textbf{Initial State:} Simulation event list is generated by initializing the simulation			
			\item \textbf{Input:} Event List of 50 Events
			\item \textbf{Description:} As the simulation is put through one step iteration on event list pops off the event list
			\item \textbf{Expected Output:} Observe that the system Event List contains 49 Events
		\end{enumerate}
\end{enumerate}

\subsection{Event List Correctness 2}
\begin{enumerate}[]
	\item \textbf{Test Factor:} Correctness
	\item \textbf{Life Cycle Phase:} Throughout Development
	\item \textbf{Type:} Functional
	\item \textbf{Dynamic/Static:} Static
	\item \textbf{Manual/Automated:} Manual
	\item \textbf{Technique:} 
		\begin{enumerate}[(i)]
			\item \textbf{Initial State:} Assume Event List is generated			
			\item \textbf{Input:} Assume event List contains 50 Events
			\item \textbf{Description:} Examine the code by working through the process of popping off one event
			\item \textbf{Expected Output:} Conclude the event list contains 49 Events
		\end{enumerate}
\end{enumerate}

\subsection{Event List Correctness 3}
\begin{enumerate}[]
	\item \textbf{Test Factor:} Correctness
	\item \textbf{Life Cycle Phase:} Throughout Development
	\item \textbf{Type:} Functional
	\item \textbf{Dynamic/Static:} Dynamic
	\item \textbf{Manual/Automated:} Automated
	\item \textbf{Technique:} 
		\begin{enumerate}[(i)]
			\item \textbf{Initial State:} Simulation event list of length 100 is generated by initializing the simulation	
			\item \textbf{Input:} Event list
			\item \textbf{Description:} As the system is tested during the course of development a report is always generated after execution
			\item \textbf{Expected Output:} System returns an output for every event that has come off the list
		\end{enumerate}
\end{enumerate}

\subsection{State Change Correctness 1}
\begin{enumerate}[]
	\item \textbf{Test Factor:} Correctness
	\item \textbf{Life Cycle Phase:} Throughout Development
	\item \textbf{Type:} Functional
	\item \textbf{Dynamic/Static:} Dynamic
	\item \textbf{Manual/Automated:} Manual
	\item \textbf{Technique:}
		\begin{enumerate}[(i)]
			\item \textbf{Initial State:} Simulation event list is generated by initializing the simulation, Porters states generated by initializing simulation and set to "Available"
			\item \textbf{Input:} Single Event to be assigned to a Porter, Porter with state "Available"
			\item \textbf{Description:} After single iteration of the simulation, execution of Event causes a single Porter to change state
			\item \textbf{Expected Output:} Porter with state "Dispatched"
		\end{enumerate}
\end{enumerate}

\subsection{State Change Correctness 2}
\begin{enumerate}[]
	\item \textbf{Test Factor:} Correctness
	\item \textbf{Life Cycle Phase:} Throughout Development
	\item \textbf{Type:} Functional
	\item \textbf{Dynamic/Static:} Static
	\item \textbf{Manual/Automated:} Manual
	\item \textbf{Technique:}
		\begin{enumerate}[(i)]
			\item \textbf{Initial State:} Assume Event list is initialized, Assume all Porters are initialized to "Available"
			\item \textbf{Input:} Assume a set of "Pending" events and a set of "Available"  
			\item \textbf{Description:} Examine the code by working through the process of single event being assigned to an "Available" Porter
			\item \textbf{Expected Output:} Conclude the Porter's state is changed to "Dispatched"
		\end{enumerate}
\end{enumerate}

\subsection{State Change Correctness 3}
\begin{enumerate}[]
	\item \textbf{Test Factor:} Correctness
	\item \textbf{Life Cycle Phase:} Throughout Development
	\item \textbf{Type:} Functional
	\item \textbf{Dynamic/Static:} Dynamic
	\item \textbf{Manual/Automated:} Automatic
	\item \textbf{Technique:}
		\begin{enumerate}[(i)]
			\item \textbf{Initial State:} Simulation event list of length 100 is generated by initializing the simulation, 100 Porter states generated by initializing simulation and set to "Available"
			\item \textbf{Input:} List Event to be assigned to a Porter, List of Porters with state "Available"
			\item \textbf{Description:} As the system is tested during the course of development a report is always generated after execution detailing the state of Porters
			\item \textbf{Expected Output:} System returns an output for every Porter state change
		\end{enumerate}
\end{enumerate}

\subsection{Porter/Event Linkage Correctness 1}
\begin{enumerate}[] 	
	\item \textbf{Test Factor:} Correctness 
	\item \textbf{Life Cycle Phase:} Throughout Development
	\item \textbf{Type:} Functional
	\item \textbf{Dynamic/Static:} Dynamic
	\item \textbf{Manual/Automated:} Manual
	\item \textbf{Technique:}
		\begin{enumerate}[(i)]
			\item \textbf{Initial State:} Simulation event list is generated by initializing the simulation, Porters states generated by initializing simulation
			\item \textbf{Input:} A single "Pending" event, Single "Available" Porter
			\item \textbf{Description:} After a single iteration the event is linked to the Porter
			\item \textbf{Expected Output:} Porter and Event a linked together uniquely by comparing to all other Porter/Event pairs
		\end{enumerate}
\end{enumerate}

\subsection{Porter/Event Linkage Correctness 2}
\begin{enumerate}[] 	
	\item \textbf{Test Factor:} Correctness 
	\item \textbf{Life Cycle Phase:} Throughout Development
	\item \textbf{Type:} Functional
	\item \textbf{Dynamic/Static:} Static
	\item \textbf{Manual/Automated:} Manual
	\item \textbf{Technique:}
		\begin{enumerate}[(i)]
			\item \textbf{Initial State:} Assume simulation event list is generated by initializing the simulation, assume porters states generated by initializing simulation
			\item \textbf{Input:} A single "Pending" event, Single "Available" Porter
			\item \textbf{Description:} Examine the code and work through the process of linking an "Pending" event to a "Available" Porter
			\item \textbf{Expected Output:} Conclude that Porter and Event a linked together uniquely by comparing to all other Porter/Event pairs
		\end{enumerate}
\end{enumerate}

\subsection{Porter/Event Linkage Correctness 3}
\begin{enumerate}[] 	
	\item \textbf{Test Factor:} Correctness 
	\item \textbf{Life Cycle Phase:} Throughout Development
	\item \textbf{Type:} Functional
	\item \textbf{Dynamic/Static:} Dynamic
	\item \textbf{Manual/Automated:} Automated
	\item \textbf{Technique:}
		\begin{enumerate}[(i)]
			\item \textbf{Initial State:} Simulation event list of length 100 is generated by initializing the simulation, 100 Porter states generated by initializing simulation
			\item \textbf{Input:} List of "Pending" Events, Set of "Available" Porters
			\item \textbf{Description:} System records every linkage created during simulation and compares them to each other.
			\item \textbf{Expected Output:} A report is generated detailing every link between Porter/Event. System compares all linkages and concludes they are unique.
		\end{enumerate}
\end{enumerate}

\subsection{Task Pool Correctness 1}
\begin{enumerate}[]
	\item \textbf{Test Factor:} Correctness
	\item \textbf{Life Cycle Phase:} Throughout Development
	\item \textbf{Type:} Functional
	\item \textbf{Dynamic/Static:} Dynamic
	\item \textbf{Manual/Automated:} Manual
	\item \textbf{Technique:}
		\begin{enumerate}[(i)]
			\item \textbf{Initial State:} Simulation event list is generated by initializing the simulation, porter states generated by initializing simulation all with states $\neq$ "Available"
			\item \textbf{Input:} Single Event, Set of Porters with state $\neq$ "Available", Empty Task Pool
			\item \textbf{Description:} Events that cannot link to a "Available" Porter go to the Task Pool
			\item \textbf{Expected Output:} Task Pool contains one additional Event 
		\end{enumerate}
\end{enumerate}

\subsection{Task Pool Correctness 2}
\begin{enumerate}[]
	\item \textbf{Test Factor:} Correctness
	\item \textbf{Life Cycle Phase:} Throughout Development
	\item \textbf{Type:} Functional
	\item \textbf{Dynamic/Static:} Static
	\item \textbf{Manual/Automated:} Manual
	\item \textbf{Technique:}
		\begin{enumerate}[(i)]
			\item \textbf{Initial State:} Assume simulation event list is initialized, assume porter states are initialized with all states $\neq$ "Available"
			\item \textbf{Input:} Single Event, Set of Porters with state $\neq$ "Available", Empty Task Pool
			\item \textbf{Description:} Examine code and work through the process of events that cannot link to a "Available" Porter, and checking that those events go to the Task Pool
			\item \textbf{Expected Output:} Conclude that the Task Pool contains one additional event 
		\end{enumerate}
\end{enumerate}

\subsection{Task Pool Correctness 3}
\begin{enumerate}[]
	\item \textbf{Test Factor:} Correctness
	\item \textbf{Life Cycle Phase:} Throughout Development
	\item \textbf{Type:} Functional
	\item \textbf{Dynamic/Static:} Dynamic
	\item \textbf{Manual/Automated:} Automatic
	\item \textbf{Technique:}
		\begin{enumerate}[(i)]
			\item \textbf{Initial State:} Simulation event list of length 100 is generated by initializing the simulation, 100 Porter states generated by initializing simulation
			\item \textbf{Input:} List of events, set of porters
			\item \textbf{Description:} Each time an event is added to the task pool the system generates a report
			\item \textbf{Expected Output:} Report of all events that were added to the Task Pool
		\end{enumerate}
\end{enumerate}

\subsection{Termination Correctness 1}
\begin{enumerate}[]
	\item \textbf{Test Factor:} Correctness
	\item \textbf{Life Cycle Phase:} Throughout Development
	\item \textbf{Type:} Functional
	\item \textbf{Dynamic/Static:} Dynamic
	\item \textbf{Manual/Automated:} Manual
	\item \textbf{Technique:}
		\begin{enumerate}[(i)]
			\item \textbf{Initial State:} Task pool is empty of events and all porters are in state "Available"  
			\item \textbf{Input:} Empty Task Pool
			\item \textbf{Description:} System recognizes initial state as a termination condition
			\item \textbf{Expected Output:} Simulation termination
		\end{enumerate}
\end{enumerate}

\subsection{Termination Correctness 2}
\begin{enumerate}[]
	\item \textbf{Test Factor:} Correctness
	\item \textbf{Life Cycle Phase:} Throughout Development
	\item \textbf{Type:} Functional
	\item \textbf{Dynamic/Static:} Static
	\item \textbf{Manual/Automated:} Manual
	\item \textbf{Technique:}
		\begin{enumerate}[(i)]
			\item \textbf{Initial State:} Assume Task pool is empty of events and assume all porters are in state "Available"  
			\item \textbf{Input:}  Assume empty Task Pool
			\item \textbf{Description:} Examine the code to conclude that a termination condition has been met 
			\item \textbf{Expected Output:} Conclude simulation termination
		\end{enumerate}
\end{enumerate}

\subsection{Termination Correctness 3}
\begin{enumerate}[]
	\item \textbf{Test Factor:} Correctness
	\item \textbf{Life Cycle Phase:} Throughout Development
	\item \textbf{Type:} Functional
	\item \textbf{Dynamic/Static:} Dynamic
	\item \textbf{Manual/Automated:} Automated
	\item \textbf{Technique:}
		\begin{enumerate}[(i)]
			\item \textbf{Initial State:} 100 simulations are initialized 
			\item \textbf{Input:} 100 simulations with termination conditions "Empty Task Pool, All Porters in state 'Available'"
			\item \textbf{Description:} Testing system starts up 100 simulations and records which have terminated once reaching the termination condition 
			\item \textbf{Expected Output:} A report is generated by a testing system detailing what systems have terminated
		\end{enumerate}
\end{enumerate}

\subsection{Input/Initialization Correctness 1}
\begin{enumerate}[]
	\item \textbf{Test Factor:} Correctness  
	\item \textbf{Life Cycle Phase:} Throughout Development
	\item \textbf{Type:} Functional
	\item \textbf{Dynamic/Static:} Dynamic
	\item \textbf{Manual/Automated:} Manual
	\item \textbf{Technique:}
		\begin{enumerate}[(i)]
			\item \textbf{Initial State:} Uninitialized Simulation   
			\item \textbf{Input:} Simulation Input File (.csv extension), containing information; 10 Porters, 10 Events
			\item \textbf{Description:} Once input file is loaded by the simulation, it has become initialized with modified values
			\item \textbf{Expected Output:} Simulation initializes 10 Porters and 10 Events
		\end{enumerate}
\end{enumerate}

\subsection{Input/Initialization Correctness 2}
\begin{enumerate}[]
	\item \textbf{Test Factor:} Correctness  
	\item \textbf{Life Cycle Phase:} Throughout Development
	\item \textbf{Type:} Functional
	\item \textbf{Dynamic/Static:} Dynamic
	\item \textbf{Manual/Automated:} Automatic
	\item \textbf{Technique:}
		\begin{enumerate}[(i)]
			\item \textbf{Initial State:} 100 Uninitialized Simulations   
			\item \textbf{Input:} 100 Simulation Input Files (.csv extension) containing information; 1 to 10 Porters,  1 to 10 Events, 100 unique combinations
			\item \textbf{Description:} Testing program takes in 100 Input files and Initializes 100 Simulation
			\item \textbf{Expected Output:} Testing Program generates a report on all 100 simulation that were initialized 
		\end{enumerate}
\end{enumerate}

\subsection{The Golden Test}
\begin{enumerate}[]
	\item \textbf{Test Factor:} Reliability \& Correctness 
	\item \textbf{Life Cycle Phase:} Throughout Development
	\item \textbf{Type:} Functional
	\item \textbf{Dynamic/Static:} Dynamic
	\item \textbf{Manual/Automated:} Manual
	\item \textbf{Technique:}
		\begin{enumerate}[(i)]
			\item \textbf{Initial State:} Uninitialized Simulation in a deterministic mode
			\item \textbf{Input:} 'Gold Copy' set of events. 5 Events; Porter moving patient from Emergency to ICU using stretcher, Porter moving oxygen tank from storage to a room in Ward A, Porter moving specimen samples from laboratory A to laboratory B, Porter moving patient from X-Ray to Ultrasound using wheelchair, Porter takes a break 
			\item \textbf{Description:} Using a 'Gold copy' of a set of events, new program builds are retested using the 'gold copy' to ensure consistency of execution by examining inconsistencies in the output file
			\item \textbf{Expected Output:} A difference file comparing 'gold copy' statistics against the newly tested code outlining the inconsistencies between them 
		\end{enumerate}
\end{enumerate}

\subsection{Usability Test 1}
\begin{enumerate}[]
	\item \textbf{Test Factor:} Ease of Use
	\item \textbf{Life Cycle Phase:} Final Stages of Development
	\item \textbf{Type:} Functional
	\item \textbf{Dynamic/Static:} Dynamic
	\item \textbf{Manual/Automated:} Manual
	\item \textbf{Technique:}
		\begin{enumerate}[(i)]
			\item \textbf{Initial State:} Simulation prior to execution  
			\item \textbf{Input:} End user with an example simulation set; Create Input file, Load Input File into Simulation, Initialize Simulation
			\item \textbf{Description:} End user is provided with a set of instructions to Initialize the Simulation 
			\item \textbf{Expected Output:} The end user successfully Initializes the Simulation
		\end{enumerate}
\end{enumerate}

\subsection{Compatibility Test}
\begin{enumerate}[] 
	\item \textbf{Test Factor:} Portability 
	\item \textbf{Life Cycle Phase:} Final Stages of Development
	\item \textbf{Type:} Functional
	\item \textbf{Dynamic/Static:} Dynamic
	\item \textbf{Manual/Automated:} Manual
	\item \textbf{Technique:}
		\begin{enumerate}[(i)]
			\item \textbf{Initial State:} Unknown Operating System with the simulation accessible on local storage  
			\item \textbf{Input:} 'Gold Copy' simulation parameters (Details outline previously under 'The Golden Test')
			\item \textbf{Description:} Simulation is run and compared to results computed on other operating systems to ensure it is functioning normally in the new environment
			\item \textbf{Expected Output:} Simulated results are consistent with previously generated values
		\end{enumerate}
\end{enumerate}

\subsection{Distribution Correctness 1}
\begin{enumerate}[]
	\item \textbf{Test Factor:} Correctness 
	\item \textbf{Life Cycle Phase:} Throughout Development
	\item \textbf{Type:} Structural
	\item \textbf{Dynamic/Static:} Dynamic
	\item \textbf{Manual/Automated:} Manual
	\item \textbf{Technique:}
		\begin{enumerate}[(i)]
			\item \textbf{Initial State:} N/A  
			\item \textbf{Input:} Known distributions in our simulation: Poisson
			\item \textbf{Description:} Run the distribution when creating the event times and examine the results through a proven program that compares the regression (Excel)
			\item \textbf{Expected Output:} After examination the results follow the Poisson distribution
		\end{enumerate}
\end{enumerate}

\section{Proof of Concept Test}
Proof of Concept Test will include these tests
\begin{enumerate}[(a)]
	\item Event List Correctness 1
	\item State Change Correctness 1
	\item Task Pool Correctness 1
	\item Termination Correctness 1
	\item Input/Initialization Correctness 1
\end{enumerate}
\section{Schedule}
\begin{center}
    \begin{tabular}{| c | l | l | p{5cm} |}
    \hline
    Phase \# & Phase Name & Start Date & Description \\ \hline
    1 & Development Phase 1 & November $1^{st}$ & N/A \\ \hline
    2 & $1^{st}$ Proof of Concept & November $12^{th}$ & 20 Minute demonstration delivered for the rest of the capstone class \\ \hline
    3 & Development Phase 2 & November $13^{th}$ & N/A \\ \hline
    4 & $2^{nd}$ Proof of Concept & December $2^{nd}$ & Demonstration delivered to the stakeholders \\ \hline
    5 & Development Phase 3 & December $3^{rd}$ & N/A \\ \hline
    6 & Test Phase 1 & March $1^{st}$ & N/A \\ \hline
    7 & Development Phase 4 & March $15^{th}$ & N/A \\ \hline
    8 & Testing Phase 2 & March $28^{th}$ & N/A \\ \hline
    9 & Final Demonstration & March $29^{th}$ & N/A \\ \hline
  	
    \end{tabular}
\end{center}


%%% End document
\end{document}