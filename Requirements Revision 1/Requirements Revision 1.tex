%%%%%%%%%%%%%%%%%%%%%%%%%%%%%%%%%%%%%%%%%%%%%%%%%%%%%%%%%%%%%%%%%%%%%%
% LaTeX Example: Project Report
%
% Source: http://www.howtotex.com
%
% Feel free to distribute this example, but please keep the referral
% to howtotex.com
% Date: March 2011 
% 
%%%%%%%%%%%%%%%%%%%%%%%%%%%%%%%%%%%%%%%%%%%%%%%%%%%%%%%%%%%%%%%%%%%%%%
% How to use writeLaTeX: 
%
% You edit the source code here on the left, and the preview on the
% right shows you the result within a few seconds.
%
% Bookmark this page and share the URL with your co-authors. They can
% edit at the same time!
%
% You can upload figures, bibliographies, custom classes and
% styles using the files menu.
%
% If you're new to LaTeX, the wikibook is a great place to start:
% http://en.wikibooks.org/wiki/LaTeX
%
%%%%%%%%%%%%%%%%%%%%%%%%%%%%%%%%%%%%%%%%%%%%%%%%%%%%%%%%%%%%%%%%%%%%%%
% Edit the title below to update the display in My Documents
%\title{Project Report}
%
%%% Preamble
\documentclass[paper=letter, fontsize=10pt]{scrartcl}
\usepackage[body={7in,7.5in},top=1in, bottom=1in]{geometry}
\usepackage[T1]{fontenc}
\usepackage{fourier}

\usepackage[english]{babel}															% English language/hyphenation
\usepackage[protrusion=true,expansion=true]{microtype}	
\usepackage{amsmath,amsfonts,amsthm} % Math packages
\usepackage[pdftex]{graphicx}	
\usepackage{url}
\usepackage{enumerate}
\usepackage{lastpage}


%%% Custom sectioning
\usepackage{sectsty}
\allsectionsfont{\normalfont\scshape}


%%% Custom headers/footers (fancyhdr package)
\usepackage{fancyhdr}
\usepackage{color,soul}
\pagestyle{fancy}
\fancyhead[L]{}
\fancyhead[c]{Requirements Revision 1}
\fancyhead[R]{\today}											
\fancyfoot[L]{}											 
\fancyfoot[C]{}											
\fancyfoot[R]{\thepage\ of \pageref{LastPage}}		% Pagenumbering
\renewcommand{\headrulewidth}{0.4pt}				% Remove header underlines
\renewcommand{\footrulewidth}{0.4pt}				% Remove footer underlines
\setlength{\headheight}{13.6pt}


%%% Equation and float numbering
\numberwithin{equation}{section}		% Equationnumbering: section.eq#
\numberwithin{figure}{section}			% Figurenumbering: section.fig#
\numberwithin{table}{section}				% Tablenumbering: section.tab#


%%% Maketitle metadata
\newcommand{\horrule}[1]{\rule{\linewidth}{#1}} 	% Horizontal rule
\newcommand{\ts}{\textsuperscript}

%%% Begin document
\begin{document}

\begin{titlepage}

\newcommand{\HRule}{\rule{\linewidth}{0.5mm}} % Defines a new command for the horizontal lines, change thickness here
\newcommand{\authors}{\shortstack{Vitaliy Kondratiev,\\Nathan Johrendt,\\Tyler Lyn,\\Mark Gammie}}

\begin{center}
 
%----------------------------------------------------------------------------------------
%	HEADING SECTIONS
%----------------------------------------------------------------------------------------

\textsc{\LARGE McMaster University}\\[1.5cm] % Name of your university/college
\textsc{\Large CAS 4ZP6}\\[0.5cm]
\textsc{\Large Team 9} \\[0.5cm]
\textsc{\Large Capstone Project 2013/2014}\\[0.5cm] % Major heading such as course name
\textsc{\large Porter Simulation}\\[0.5cm] % Minor heading such as course title

%----------------------------------------------------------------------------------------
%	TITLE SECTION
%----------------------------------------------------------------------------------------

\HRule \\[0.4cm]
{ \huge \bfseries Requirements Revision 1}\\[0.4cm] % Title of your document
\HRule \\[1.5cm]
 
%----------------------------------------------------------------------------------------
%	AUTHOR SECTION
%----------------------------------------------------------------------------------------

\begin{minipage}{0.4\textwidth}
\begin{flushleft} \large
\emph{Authors:}\\
Vitaliy Kondratiev - 0945220\\
Nathan Johrendt - 0950519\\
Tyler Lyn - 0948978\\
Mark Gammie - 0964156
\end{flushleft}
\end{minipage}
~
\begin{minipage}{0.4\textwidth}
\begin{flushright} \large
\emph{Supervisor:} \\
Dr. Douglas Down % Supervisor's Name
\end{flushright}
\end{minipage}\\[4cm]

% If you don't want a supervisor, uncomment the two lines below and remove the section above
%\Large \emph{Author:}\\
%John \textsc{Smith}\\[3cm] % Your name

%----------------------------------------------------------------------------------------
%	DATE SECTION
%----------------------------------------------------------------------------------------

{\large \today}\\[3cm] % Date, change the \today to a set date if you want to be precise

%----------------------------------------------------------------------------------------
%	LOGO SECTION
%----------------------------------------------------------------------------------------

%\includegraphics{Logo}\\[1cm] % Include a department/university logo - this will require the graphicx package
 
%----------------------------------------------------------------------------------------
%Template taken from: http://www.softwaretestinghelp.com/test-plan-sample-softwaretesting-and-quality-assurance-templates/

\vfill % Fill the rest of the page with whitespace
\end{center}
\end{titlepage}

\setcounter{tocdepth}{2}

\tableofcontents

\newpage
\section{Revision History}
\begin{center}
    \begin{tabular}{| c | l | l | l |}
    \hline
    Revision \# & Author & Date & Comment \\ \hline
  	1 & \shortstack{\\Vitaliy Kondratiev,\\Nathan Johrendt,\\Tyler Lyn,\\Mark Gammie} & October 28 & Info Missing \\ \hline
  	2 & \shortstack{\\Vitaliy Kondratiev,\\Nathan Johrendt,\\Tyler Lyn,\\Mark Gammie} & October 29 & Info Missing \\ \hline
  	3 & \shortstack{\\Vitaliy Kondratiev,\\Nathan Johrendt,\\Tyler Lyn,\\Mark Gammie} & October 29 & Info Missing \\ \hline
  	4 & \shortstack{\\Vitaliy Kondratiev,\\Nathan Johrendt,\\Tyler Lyn,\\Mark Gammie} & October 30 & Info Missing \\ \hline
  	5 & \shortstack{\\Nathan Johrendt} & January 13 & Info Update \\ \hline
  	6 & \shortstack{\\Vitaliy Kondratiev} & February 2 & Update and Corrections \\ \hline
  	7 & \shortstack{\\Vitaliy Kondratiev} & April 11 & Update and Corrections \\ \hline
  	8 & \shortstack{\\Nathan Johrendt} & April 15 & Update and Corrections \\ 
    \hline
    \end{tabular}
\end{center}

\section{Purpose of the Project}
\subsection{Template}
This requirements document is based on the Volere template, formatted using LaTex.
\subsection{Background}
Hamilton Health Sciences have become aware of several inefficiencies in their porter services at their Juravinski Hospital location. Porter services, in this context, are defined as the movement of hospital equipment and patient transfers from one area to another. Porters are a key  piece of overall patient experience and satisfaction; the flow of day to day operations in a hospital depends on their efficiency. Some of the particular issues they identified relate to porters complying with operational policies, as well as finding ways to better handle spikes in work volume on busy days. Hamilton Health Sciences have gathered substantial operational data on porter activity, but are lacking the tools to interpret it.
\subsection{Goals}
Our goal is to provide HHS with the tools to simulate their porter services so that they can test their own solutions, methods and make calculated decisions based on the results.

\section{The Stakeholders-Clients-Customers}
Hamilton Health Sciences (HHS) operational management team is the main stakeholder/client/customer for this project. Names and position are: \\
Corey Stark (CSS - Sodexo Systems and Performance Manager)\\
Kym Kempf (Business and Program Manager - Corporate Services)\\
David DiSimoni (Site Manager - Customer Support Services)\\
Anita Lamond (Director - Corporate Services)\\
Steve Metham (Manager - Quality)\\
Mohammad Majedi (Quality Specialist)\\
Talha Hussain (Quality Specialist)
\subsection{Other Stakeholders}
Patients and Hospital Staff are the secondary stakeholders for this project. Any benefits that arise from the successful completion of this project will affect these stakeholders.
\subsection{Hands on Users}
Operational Management Staff will be the primary hands on users of the finished product. Names and positions are: \\
Talha Hussain (Quality Specialist)\\
Corey Stark (CSS - Sodexo Systems and Performance Manager)

\section{Mandated Constraints}
\subsection{Solution Constraints}
Given that this software is an imitation of real-world events at an HHS hospital, and that real historical data is used by the simulation, the results produced must be accurate and consistent.
\subsection{Schedule Constraints}
Simulation Software must be completed and requirements met by the end of April 2014.  

\section{Naming Conventions and Technology}
\subsection{Definitions of All Terms, Including Acronyms, Used by Stakeholders involved in the Project}
\begin{enumerate}[(a)]
	\item \textbf{HHS:} Hamilton Health Sciences
	\item \textbf{IVR:} Interactive Voice Request - phone system for requesting porter services
	\item \textbf{Porter:} Staff member responsible for movement of equipment such as beds, wheelchairs, other medical instruments and patient transfers from one location to another
	\begin{enumerate}[(i)]
		\item \textbf{Off-System Porter:} Porters that follow a strict scheduled and a predetermined set of activities
		\item \textbf{On-System Porter:} Porters that respond to ad-hoc and pre-booked requests	
	\end{enumerate}
	\item \textbf{Dispatching System:} An automated software system responsible for receiving and assigning requests to Porters
	\item \textbf{Standard Equipment:} Non-powered stretchers, beds, wheelchairs
	\item \textbf{Priority of Requests:} Requests placed by Hospital Staff can be prioritized on a scale from 0 - 9 with 0 being the most urgent. Porters can place an Assist Call that has a higher priority than 0.
	\item \textbf{Event State:} A state of the porter service event as dictated by the Dispatching System
	\begin{enumerate}[(i)]
		\item \textbf{Pending:} Job has been placed in the system queue
		\item \textbf{Dispatched:} Job has been matched to an available porter
		\item \textbf{In-Progress:} Job is being executed by the porter
		\item \textbf{Complete:} Job has been completed
		\item \textbf{Dispatch Delay:} Porter states that he/she is delayed during a Dispatched event
		\item \textbf{In-Progress Delay:} Porter states that he/she is delayed during a In-Progress		
	\end{enumerate} 
	\item \textbf{Transaction Time:} the time from Event State (Pending) to Event State (Complete)
	\item \textbf{Proactive Page:} A porter pages the request location to inform the requester of his/her impending arrival      
	\item \textbf{Age of Request:} How long a job has been pending in the dispatch system
	 do not receive any calls from the dispatch system
	\item \textbf{CSV file:} Comma Separated Values file, stores tabular data in plain text form. 
	\item \textbf{GUI:} Graphical User Interface - interaction with electronic devices through graphical icons and visual indicators
	\item \textbf{Dashboard:} Visualization of the data formatted to serve a purpose in critical decision making
\end{enumerate}

\section{Relevant Facts and Assumptions}
\subsection{Relevant Facts}
\begin{enumerate}[(a)]
	\item HHS will provide the project team with available non critical data
	\item HHS currently uses a Dispatching System to route its porters to desired locations
	\item On certain route segments, two porters are required to transport the patient 
	\item HHS will provide personnel and hours for testing of the application
	\item HHS will be available to answer any queries as well as give feedback on the ongoing project milestones
\end{enumerate}
\subsection{Business Rules}
\begin{enumerate}[(a)] 
	\item Each request has six event states (Pending, Dispatch, In-Progress, Complete, Dispatch Delay, In-Progress Delay)
	\item Requests are prioritized on a 0 - 9 scale
	\item There are two types of porters (On-System, Off-System)
	\item Every completed event has an associated transaction time, unless cancelled
	\item Dispatch System determines the assignment of requests by using these parameters (Priority, Proximity, Pre-Scheduled Appointment Use, Age of Request) 
	\item Industry standard for patient transport transaction time is 30 minutes
	\item Porter service requests can be made using any hospital computer or phone (IVR)
	\item Porters can be "zoned" into system item requests by the dispatching system  
\end{enumerate}	    
\subsection{Assumptions}
\begin{enumerate}[(a)]
	\item Every porter is equally capable of performing every task as every other porter
	\item Some of the porters use proactive paging  
	\item The majority of service requests are made through hospital computers
	\item Data is 100 percent accurate but may not represent the each situation exactly, as porters do not always follow operational policies 
\end{enumerate}

\section{Scope of the Work}
\subsection{Current Situation}
Hamilton Health Sciences are experiencing inefficiencies when synchronizing their porter services throughout each of their locations and are lacking the tools to solve this problem. The biggest problem comes from the lack of compliance and coordination of the many separate entities of the hospital body. The porters are currently being scheduled by an online dispatching system, which also tracks their progress. Although the system is very efficient at how it completes it's dispatching, it has no insight or analysis capabilities to review past recorded data. 
\subsection{Context of the Work}
HHS requires a tool to support their daily operations and decision making. The tool is to provide the stakeholders with the data and insight to complete their objectives.
\subsection{Business Use Case}
The simulation tool will be used by members of the operational management staff to trial new scheduling, compliance and overflow handling techniques in a no cost environment. Any results can then be used to reinforce arguments for making changes around the hospital.

\section{Scope of the Product}
\subsection{Product Boundary}
\begin{enumerate}[(a)]
	\item The simulation tool will not model the 100\% full hospital environment
	\item The simulation will only consider the On-System Porters
	\item Not all porter activities will be simulated. Simulation will concentrate on the 6 event states tracked by the dispatching system (Pending, Dispatch, In-Progress, Complete, Dispatch Delay, In-Progress Delay).
	\item Simulation will be limited to seven days of modelling
	\item Main focus of output analysis will be porter wait times, transaction time, numbers of completed and cancelled jobs
\end{enumerate} 
\subsection{Product Use Cases}
\begin{enumerate}[(a)]
	\item Operational Manager has a new initiative they want to implement into everyday operation. He/She uses the simulation by changing the adjustable variables with his/her own values and executing it. He/She analyses the output of the simulation and determines if the new initiative should be implemented.
	\item Operational Manager has to determine how to modify the schedule for the porter service staff. He/She uses the simulation by changing the adjustable variables with his/her own values, importing a test version of the schedule and executing it. He/She uses the output to design/refine the new schedule.
	\item Operational Manager wants to increase operational compliance of some particular policy. He/She uses the simulation by changing the adjustable variables related to a certain level of compliance with his/her own values and executing it. Once positive results have been verified he/she shows the results to all the parties involved in the compliance policy to effectively increase compliance.
	\item Operational Manager wants to experiment with theoretical scenarios. He/She uses the simulation by changing the adjustable variables with his/her own values and executing it. He/She analyses the output data and either creates a new initiative based on result or archives the data.     
\end{enumerate}

\section{Functional Requirements}
\subsection{Functional Requirements}
\begin{enumerate}
	\item \textbf{Description:} Simulation must take a file as input. This file contains data logs from the dispatching system concerning past porter events
	\\ \textbf{Rationale:} Operational Management staff has indicated that using a file as input is the preferred method 
	\\ \textbf{Fit Criterion:} Simulation must accept the file without error 100\% of the time
	\item \textbf{Description:} A series of simulation variables that affect the simulation output must be editable by the user
	\\ \textbf{Rationale:} Operational Management staff must be able to modify the simulation
	\\ \textbf{Fit Criterion:} Simulation must include the following variables   
	\begin{enumerate}[(i)]
		\item Simulation Duration
		\item Porter Wait Times
		\item Job Flow
		\item Start Day
		\item Appointment Factor
		\item Automatic Job Priority Values
		\item Weighted Job List
		\item Random Seed
	\end{enumerate}
	\item \textbf{Description:} Simulation Tool must be able to run a pre-designed model incorporating the given input variables and terminate correctly
	\\ \textbf{Rationale:} Operational Staff must be able to run the simulation
	\\ \textbf{Fit Criterion:} The simulation will terminate to produce output results
	\item \textbf{Description} The output of the simulation must be relevant data
	\\ \textbf{Rationale:} Data must be relevant for the Operational Management's business process
	\\ \textbf{Fit Criterion:} The output must be 100\% relevant in the scope of the problem
	\item \textbf{Description} The output of the simulation must contain a visual dashboard
	\\ \textbf{Rationale:} The data needs to be aggregated and presented to the user
	\\ \textbf{Fit Criterion:} The output dashboard must be have 0\% errors
	\item \textbf{Description} The simulation must have a GUI
	\\ \textbf{Rationale:} The user must be able to use the simulation tool intuitively  
	\\ \textbf{Fit Criterion:} The user will know how to use the tool with minimal training
\end{enumerate}

\section{Look and Feel Requirements}
\subsection{Appearance Requirements}
\begin{enumerate}
	\item The simulation is required to contain output graphs in an excel dashboard. The specific metrics and design of each graph have been refined by stakeholders and are open to modification at their preference.
\end{enumerate}
\subsection{Style Requirements}
\begin{enumerate}
	\item Software must contain elements of basic human/computer interface design as expected by a casual user of personal computers and popular software/operating systems.
\end{enumerate}

\section{Usability and Humanity Requirements}
\subsection{Ease of Use Requirements}
\begin{enumerate}
	\item \textbf{Content:} Software must have a GUI
	\\	  \textbf{Motivation:} Users of this software are not assumed to be advanced computer users. Users are not expected to know how to use command line or similar advanced interfaces. 
	\\	  \textbf{Fit Criterion:} All simulation variables of the software are accessible through a GUI
	\\	  \textbf{Considerations:} This Ease of Use requirement considers all of the Product Use Cases
	\item \textbf{Content:} The GUI must have checks in place to prevent the user from using invalid inputs 
	\\	  \textbf{Motivation:} All inputs must comply with the arguments of the execution program
	\\	  \textbf{Fit Criterion:} GUI restricts the user to a predetermined set of inputs
	\\	  \textbf{Considerations:} This Ease of Use requirement considers all of the Product Use Cases
	\item \textbf{Content:} Software must be easy to navigate
	\\	  \textbf{Motivation:} Users should be able to easily move between different screens
	\\	  \textbf{Fit Criterion:} Each screen is linked to each other with an easily accessible interface feature
	\\	  \textbf{Considerations:} This Ease of Use requirement considers all of the Product Use Cases
	\item \textbf{Content:} User must clearly understand all the functions with minimal training
	\\	  \textbf{Motivation:} User should be able to pick up the functionality based on the context material
	\\	  \textbf{Fit Criterion:} All elements of GUI will be easy to understand under context of the usability
	\\	  \textbf{Consideration:} This Ease of Use requirement considers all of the Product Use Cases
\end{enumerate}
\subsection{Learning Requirements}
\begin{enumerate}
	\item \textbf{Content:} Software must be easy to learn with some hands-on training and documentation by a casual user of personal computers
	\\	  \textbf{Motivation:} Users are not required to have any knowledge of simulation software to operate the product
	\\	  \textbf{Fit Criterion:} Users will be able to use the software after two or three training sessions of less than sixty minutes
	\\	  \textbf{Consideration:} This Learning requirement considers all of the Product Use Cases
\end{enumerate}
\subsection{Understandability and Politeness Requirements}
\begin{enumerate}
	\item \textbf{Content:} Users should be able to quickly understand how the software will benefit them in their business process
	\\	  \textbf{Motivation:} Users are not expected to understand aspects that do not directly relate to their purpose
	\\	  \textbf{Fit Criterion:} All of the simulated aspects will be related to the user's business problems unless the case considered is far out of the problem scope stated in these requirements
	\\	  \textbf{Consideration:} This Understandability requirement considers all of the Product Use Cases	
\end{enumerate}

\section{Performance Requirements}
\subsection{Speed and Latency Requirements}
\begin{enumerate}
	\item \textbf{Content:} Software must be able to complete the simulation as set up by the user within a reasonable time
	\\	  \textbf{Motivation:} As per request by the stakeholders
	\\	  \textbf{Fit Criterion:} A single simulation should not take more than five minutes to complete
	\\	  \textbf{Considerations:} This speed requirement considers all of the Product Use Cases
\end{enumerate}
\subsection{Precision or Accuracy Requirements}
\begin{enumerate}
	\item \textbf{Content:} Simulation must be 100\% precise to the data source being sampled, even if the supplied data is not 100\% precise 
	\\	  \textbf{Motivation:} Software cannot ensure that every data entry recorded is valid, but can filter out issues through parsing
	\\	  \textbf{Fit Criterion:} Only valid jobs with complete data entries will be processed by the simulation
	\\	  \textbf{Considerations:} This precision requirement considers all of the Product Use Cases
\end{enumerate}
\subsection{Reliability and Availability Requirements}
	\begin{enumerate}
		\item \textbf{Content:} Software must output relevant data to the user without error
		\\	  \textbf{Motivation:} Users should expect the output to be useful in their business process
		\\	  \textbf{Fit Criterion:} Output will be in correct format as per Functional Requirement \# 4
		\\	  \textbf{Considerations:} This reliability requirement considers all of the Product Use Cases 
		\item \textbf{Content:} Software must be available to the user at all times except when a simulation is running
		\\	  \textbf{Motivation:} Users should be able to access and use the software at any point in time
		\\	  \textbf{Fit Criterion:} Software is available to use 100\% of the time other than when the simulation is executing
		\\	  \textbf{Considerations:} This availability requirement considers all of the Product Use Cases 
	\end{enumerate}
\subsection{Robustness or Fault-Tolerance Requirements}
	\begin{enumerate}
		\item \textbf{Content:} Software must not crash during the simulation process if the simulation is running within the scope of the project
		\\	  \textbf{Motivation:} Users should expect most simulations to complete without error
		\\	  \textbf{Fit Criterion:} The simulation should not fail 99\% of the time
		\\	  \textbf{Considerations:} This robustness requirement considers all of the Product Use Cases
		\item \textbf{Content:} In the event of failure the user will be made aware the failure event and reason
		\\	  \textbf{Motivation:} Users should expect feedback on failure
		\\	  \textbf{Fit Criterion:} Simulation gives feedback on failure 100\% of the time unless a failure event has crashed the software as a whole
		\\	  \textbf{Considerations:} This robustness requirement considers all of the Product Use Cases
	\end{enumerate}
\subsection{Capacity Requirements}
\begin{enumerate}
		\item \textbf{Content:} Software must be capable of simulating a large number of porters
		\\	  \textbf{Motivation:} Users should expect to schedule as many porters as they would like to test theories, any more than
		\\	  \textbf{Fit Criterion:} Users are not met with any bounds, beyond system memory limits, when specifying porter numbers
		\\	  \textbf{Considerations:} This reliability requirement considers all of the Product Use Cases 
		\item \textbf{Content:} Software must be able to read very large quantities of input data
		\\	  \textbf{Motivation:} Users should not be bounded by how much input data can be fed into the simulation
		\\	  \textbf{Fit Criterion:} Users are not met with any bounds when specifying input data, beyond system memory limits
		\\	  \textbf{Considerations:} This reliability requirement considers all of the Product Use Cases 
	\end{enumerate}

\section{Operational and Environmental Requirements}
\subsection{Expected Physical Environment}
Software on a computer station or terminal in an HHS employee's office.
\subsection{Release Requirements}
Final version of simulation software should be made available by April 20\ts{th} 2014.

\section{Maintainability and Support Requirements}
\subsection{Maintenance Requirements}
	\begin{enumerate}
		\item The Project Team will provide maintenance to the software up to the projected project finish date (April 29\ts{th} 2014)
	\end{enumerate}
\subsection{Supportability Requirements}
	\begin{enumerate}
		\item The Project Team will provide support for the software up to the projected project finish date (April 29\ts{th} 2014)
	\end{enumerate}
\subsection{Adaptability Requirements}
	\begin{enumerate}
		\item The simulation is being modelled after existing data provided by stakeholders, with the possibility of modifying the software later to accommodate new hospital locations or additional  
	\end{enumerate}

\section{Security Requirements}
\subsection{Access Requirements}
	\begin{enumerate}
		\item Software will be accessible to any user who has access to the system the Porter Simulation is stored on
	\end{enumerate}
\subsection{Privacy Requirements}
	\begin{enumerate}
		\item Confidentiality waivers are required for project members to participate in on-site visits to HHS locations during simulation development.
	\end{enumerate}

\section{Off-the-Shelf Solutions}
\subsection{Ready-Made Products}
Visual8 produces visual process modelling simulations and have worked with HHS on past projects.
\subsection{Reusable Components}
Simulation model should be adaptable to other HHS locations utilizing on-system porter monitoring and logging services.
\subsection{Products That Can Be Copied}
None found that are applicable and freely available to duplicate.

\section{New Problems}
\subsection{Effects on the Current Environment}
Only when the product simulates positive beneficial results consistently will stakeholders consider implementing modifications to the existing HHS environment.
\subsection{Effects on the Installed Systems}
The product will have 0\% effect on installed systems or other software.
\subsection{Limitations in the Anticipated Implementation Environment That May Inhibit the New Product}
The simulation software should only be executed on systems that meet the previously stated Performance Requirements.
\subsection{Follow-Up Problems}
Should major changes in HHS operational protocol occur in the future, aspects of the simulation will likely require modification to continue producing accurate results.

\section{Tasks}
\subsection{Stakeholder Milestones}
\begin{enumerate}[(a)]	
	\item \textbf{Deliver first demo to HHS representatives} - due by February 4\ts{th}, 2014
	\item \textbf{Deliver second demo of the product to HHS staff} - due by March 21\ts{st}, 2014
	\item \textbf{Lead usability tests with users of the product} - due by March 28\ts{th}, 2014
	\item \textbf{Provide final project report + design documentation + user manual + Final Demonstration} - due by April 16\ts{th}, 2014
	\item \textbf{Install final version of software on department computers} - due by April 16\ts{th}, 2014
\end{enumerate}
\section{Risks}
\begin{enumerate}[(a)]
	\item Simulation gives false data leading to wrong decisions by the Operational Management Team
	\item Simulation is used as not intended/ out of scope leading to wrong decisions by the Operational Management Team
	\item Simulation does not help with decision making process of the Operational Management Team 
\end{enumerate}
\section{Costs}
There are currently no financial costs associated with this project.

\section{User Documentation and Training}
\subsection{User Documentation Requirements}
The users of this software will be provided with detailed documentation outlining the framework, functionality, and usability
\subsection{Training Requirements}
The users will be provided with hands-on training and training material by the project team.

\section{Open Issues}
\begin{enumerate}
	\item Output graphs need further refinement for consistency 
	\item Further validation and verification of simulated results would provide additional opportunity for refinement 
\end{enumerate}

\section{Waiting Room}
\begin{enumerate}
	\item Further development into a flexible web application
	\item Remove dependency on excel by graphing through python libraries 
\end{enumerate}

\section{List of Tables}
\begin{enumerate}[(a)]
	\item Revision History Table - Section 1 contains a table detailing the revision history of the document.
\end{enumerate}

%%% End document
\end{document}